%% RiSE Latex Template - version 0.5
%%
%% RiSE's latex template for thesis and dissertations
%% http://risetemplate.sourceforge.net
%%
%% (c) 2012 Yguaratã Cerqueira Cavalcanti (yguarata@gmail.com)
%%          Vinicius Cardoso Garcia (vinicius.garcia@gmail.com)
%%
%% This document was initially based on UFPEThesis template, from Paulo Gustavo
%% S. Fonseca.
%%
%% ACKNOWLEDGEMENTS
%%
%% We would like to thanks the RiSE's researchers community, the 
%% students from Federal University of Pernambuco, and other users that have
%% been contributing to this projects with comments and patches.
%%
%% GENERAL INSTRUCTIONS
%%
%% We strongly recommend you to compile your documents using pdflatex command.
%% It is also recommend use the texlipse plugin for Eclipse to edit your documents.
%%
%% Options for \documentclass command:
%%         * Idiom
%%           pt   - Portguese (default)
%%           en   - English
%%
%%         * Text type
%%           bsc  - B.Sc. Thesis
%%           msc  - M.Sc. Thesis (default)
%%           qual - PHD qualification (not tested yet)
%%           prop - PHD proposal (not tested yet)
%%           phd  - PHD thesis
%%
%%         * Media
%%           scr  - to eletronic version (PDF) / see the users guide
%%
%%         * Pagination
%%           oneside - unique face press
%%           twoside - two faces press
%%
%%		   * Line spacing
%%           singlespacing  - the same as using \linespread{1}
%%           onehalfspacing - the same as using \linespread{1.3}
%%           doublespacing  - the same as using \linespread{1.6}
%%
%% Reference commands. Use the following commands to make references in your
%% text:
%%          \figref  -- for Figure reference
%%          \tabref  -- for Table reference
%%          \eqnref  -- for equation reference
%%          \chapref -- for chapter reference
%%          \secref  -- for section reference
%%          \appref  -- for appendix reference
%%          \axiref  -- for axiom reference
%%          \conjref -- for conjecture reference
%%          \defref  -- for definition reference
%%          \lemref  -- for lemma reference
%%          \theoref -- for theorem reference
%%          \corref  -- for corollary reference
%%          \propref -- for proprosition reference
%%          \pgref   -- for page reference
%%
%%          Example: See \chapref{chap:introduction}. It will produce 
%%                   'See Chapter 1', in case of English language.

\documentclass[en,twoside,onehalfspacing,msc]{risethesis}

\usepackage{natbib}
\usepackage{babel}
\usepackage{supertabular}
\usepackage{microtype}

%% Change the following pdf author attribute name to your name.
\usepackage[linkcolor=blue,citecolor=blue,urlcolor=blue,colorlinks,pdfpagelabels,pdftitle={rise-template-msc},pdfauthor={RiSE Latex Template}]{hyperref}

\address{RECIFE}

\universitypt{Universidade Federal de Pernambuco}
\universityen{Federal University of Pernambuco}

\departmentpt{Centro de Informática}
\departmenten{Center for Informatics}

\programpt{Pós-graduação em Ciência da Computação}
\programen{Graduate in Computer Science}

\majorfieldpt{Ciência da Computação}
\majorfielden{Computer Science}

\title{A Bug Report Analysis and Search Tool}
\date{JULY/2009}

\author{Yguaratã Cerqueira Cavalcanti}
\adviser{Silvio Romero de Lemos Meira}
\coadviser{Eduardo Santana de Almeida}

\begin{document}

\frontmatter
\frontpage
\presentationpage

\begin{dedicatory}
I dedicate this dissertation to myself and all my family, friends and
professors who gave me all necessary support to get here.
\end{dedicatory}

\acknowledgements
I would like to thank and dedicate this dissertation to the following people:
\ldots

\begin{epigraph}[Open My Eyes]{S.O.J.A.}
I open my eyes each morning I rise, to find a true\\
thought, know that it's real, I'm lucky to breathe,\\
I'm lucky to feel, I'm glad to wake up, I'm glad to be\\
here, with all of this world, and all of it's pain, all\\
of it's lies, and all of it's flipped down, I still\\
feel a sense of freedom, so glad I'm around,\\
\vspace{0.5cm}
It's my freedom, can't take it from me, i know it, it\\
won't change, but we need some understanding, I know\\
we'll be all right.
\end{epigraph}

\resumo
% Escreva seu resumo no arquivo resumo.tex
O gerenciamento eficiente de solicitações de mudança (SM) é fundamental para o
sucesso das atividades de manutenção e evolução de software. Entretanto, a
atribuição de SMs a desenvolvedores de software é um aspecto custoso desse
gerenciamento, pois demanda tempo e requer conhecimento apropriado do projeto de
software. Com o propósito de diminuir esse custo, várias pesquisas já propuseram
métodos de atribuição automática de SMs. Embora representem avanços na área,
existem vários fatores inerentes a atribuição de SMs que não são considerados
nessas pesquisas e são essenciais para a automação.

Como demonstrado nesse trabalho, a atribuição automática deve, por exemplo,
considerar a carga de trabalho, a experiência e o conhecimento dos
desenvolvedores, a prioridade e a severidade das SMs, a afinidade dos
desenvolvedores com os problemas descritos nas SMs, e até mesmo os
relacionamentos interpessoais. Para tornar esse cenário ainda mais complexo,
esses fatos podem variar de acordo com o projeto de software que está sendo
desenvolvido. Assim, uma solução para o problema de atribuição de SMs depende de
informações contextuais.

Assim, esse trabalho propõe, implementa e valida uma solução arquitetural
sensível ao contexto para atribuição automática de SMs. Dado o aspecto
contextual da solução, a arquitetura enfatiza a necessidade de considerar as
diversas fontes de informações presentes na organização, assim como a
necessidade de se desenvolver algorítimos que implementem diferentes estratégias
de atribuição. A proposta e implementação dessa solução é embasada em resultados
de duas pesquisas quantitativas: um estudo de mapeamento sistemático da
literatura, e uma pesquisa de questionário com desenvolvedores de software. Esse
último forneceu um conjunto de requisitos que a solução automatizada deve
satisfazer para que as estratégias de atribuição sejam atendidas, enquanto o
mapeamento da literatura identificou técnicas, algoritmos, e outros requisitos
necessários a automação.

A implementação da arquitetura segue uma estratégia de automação, também
elabo\-rada nesse trabalho, que possui dois componentes principais: um sistema
especialista baseado em regras (SEBR); e um modelo de recuperação de informação
(MRI) com técnicas de aprendizagem. Em nossa estratégia, esses dois componentes
são executados alternadamente em momentos diferentes a fim de atribuir uma SM
automaticamente. O SEBR processa regras simples e complexas, considerando
informações contextuais do projeto de software e da organização que o
desenvolve. O MRI é utilizado para fazer o casamento entre SMs e desenvolvedores
de acordo com o histórico de atribuições.

\begin{keywords}
Engenharia de Software, Manutenção e Evolução de Software, Gerenciamento de
Solicitações de Mudança, Atribuição Automática de Solicitações de Mudança
\end{keywords}

\abstract
% Write your abstract in a file called abstract.tex
Software maintenance and evolution are characterised by their huge cost and
slow speed of implementation. Yet they are inevitable activities -- almost all
software that is useful and successful stimulates user-generated requests for
change and improvements. \citeauthor{Sommerville2007} is
even more emphatic and says that software changes is a fact of life for large
software systems. In addition, a set of studies has stated along the years that
software maintenance and evolution is the most expensive phase of software
development, taking up to 90\% of the total costs.

All those characteristics from software maintenance lead the academia and
industry to constantly investigate new solutions to reduce costs in such
phase. In this context, Software Configuration Management (SCM) is a set of
activities and standards for managing and evolving software; SCM defines how
to record and process proposed system changes, how to relate these to system
components, among other procedures. For all these tasks it has been proposed
different tools, such as version control systems and bug trackers. However, some
issues may arise due to these tools usage, such as the dynamic assignment of a
developer to a bug report or the bug report duplication problem.

In this sense, this dissertation investigates the problem of bug report
duplication emerged by the use of bug trackers on software development
projects. The problem of bug report duplication is characterized by the
submission of two or more bug reports that describe the same software issue, and
the main consequence of this problem is the overhead of rework when managing
these bug reports.

\begin{keywords}
bug reports, bug trackers, bug report duplication, change request,
tool experiment, bug report duplication characterization study, bug report
search and analysis tool
\end{keywords}

% Summary (tables of contents)
\tableofcontents

% List of figures
\listoffigures

% List of tables
\listoftables

% List of acronyms
% Acronyms manual: http://linorg.usp.br/CTAN/macros/latex/contrib/acronym/acronym.pdf
\listofacronyms
\chapter*{List of Acronyms}
\addcontentsline{toc}{chapter}{List of Acronyms}
\begin{acronym}[C.E.S.A.R.]
  \acro{AJAX}{Asynchronous JavaScript and XML}
  \acro{BAST}{Bug Report Analysis and Search Tool}
  \acro{BTT}{Bug Report Tracker Tool}
  \acro{BRN}{Bug Report Network}
  \acro{CCB}{Change Control Board}
  \acro{C.E.S.A.R.}{Recife Center For Advanced Studies and Systems
  \acroextra{C.E.S.A.R. (\url{http://www.cesar.org.br}) is a CMMi level 3
  company with around 700 employees}}
  \acro{FR}{Functional Requirement}
  \acro{GQM}{Goal Question Metric}
  \acro{LOC}{Lines of Code}
  \acro{NFR}{Non-Functional Requirement}
  \acro{NLP}{Natural Language Processing}
  \acro{ORM}{Object-Relational Mapper}
  \acro{RiSE}{Reuse in Software Engineering Group \acroextra{\url{http://www.rise.com.br}}}
  \acro{SCM}{Software Configuration Management}
  \acro{SD}{Standard Deviation}
  \acro{TF-IDF}{Term Frequency-Inverse Document Frequency}
  \acro{UFPE}{Federal University of Pernambuco}
  \acro{VSM}{Vector Space Model}
  \acro{WAD}{Work as Design}
  \acro{XP}{eXtreme Programming}
\end{acronym}

% List of listings
\lstlistoflistings

\mainmatter

%% RiSE Latex Template - version 0.5
%%
%% RiSE's latex template for thesis and dissertations
%% http://risetemplate.sourceforge.net
%%
%% (c) 2012 Yguaratã Cerqueira Cavalcanti (yguarata@gmail.com)
%%          Vinicius Cardoso Garcia (vinicius.garcia@gmail.com)
%%
%% This document was initially based on UFPEThesis template, from Paulo Gustavo
%% S. Fonseca.
%%
%% ACKNOWLEDGEMENTS
%%
%% We would like to thanks the RiSE's researchers community, the 
%% students from Federal University of Pernambuco, and other users that have
%% been contributing to this projects with comments and patches.
%%
%% GENERAL INSTRUCTIONS
%%
%% We strongly recommend you to compile your documents using pdflatex command.
%% It is also recommend use the texlipse plugin for Eclipse to edit your documents.
%%
%% Options for \documentclass command:
%%         * Idiom
%%           pt   - Portguese (default)
%%           en   - English
%%
%%         * Text type
%%           bsc  - B.Sc. Thesis
%%           msc  - M.Sc. Thesis (default)
%%           qual - PHD qualification (not tested yet)
%%           prop - PHD proposal (not tested yet)
%%           phd  - PHD thesis
%%
%%         * Media
%%           scr  - to eletronic version (PDF) / see the users guide
%%
%%         * Pagination
%%           oneside - unique face press
%%           twoside - two faces press
%%
%%		   * Line spacing
%%           singlespacing  - the same as using \linespread{1}
%%           onehalfspacing - the same as using \linespread{1.3}
%%           doublespacing  - the same as using \linespread{1.6}
%%
%% Reference commands. Use the following commands to make references in your
%% text:
%%          \figref  -- for Figure reference
%%          \tabref  -- for Table reference
%%          \eqnref  -- for equation reference
%%          \chapref -- for chapter reference
%%          \secref  -- for section reference
%%          \appref  -- for appendix reference
%%          \axiref  -- for axiom reference
%%          \conjref -- for conjecture reference
%%          \defref  -- for definition reference
%%          \lemref  -- for lemma reference
%%          \theoref -- for theorem reference
%%          \corref  -- for corollary reference
%%          \propref -- for proprosition reference
%%          \pgref   -- for page reference
%%
%%          Example: See \chapref{chap:introduction}. It will produce 
%%                   'See Chapter 1', in case of English language.

\documentclass[en,twoside,onehalfspacing,msc]{risethesis}

\usepackage{natbib}
\usepackage{babel}
\usepackage{supertabular}
\usepackage{microtype}

%% Change the following pdf author attribute name to your name.
\usepackage[linkcolor=blue,citecolor=blue,urlcolor=blue,colorlinks,pdfpagelabels,pdftitle={rise-template-msc},pdfauthor={RiSE Latex Template}]{hyperref}

\address{RECIFE}

\universitypt{Universidade Federal de Pernambuco}
\universityen{Federal University of Pernambuco}

\departmentpt{Centro de Informática}
\departmenten{Center for Informatics}

\programpt{Pós-graduação em Ciência da Computação}
\programen{Graduate in Computer Science}

\majorfieldpt{Ciência da Computação}
\majorfielden{Computer Science}

\title{A Bug Report Analysis and Search Tool}
\date{JULY/2009}

\author{Yguaratã Cerqueira Cavalcanti}
\adviser{Silvio Romero de Lemos Meira}
\coadviser{Eduardo Santana de Almeida}

\begin{document}

\frontmatter
\frontpage
\presentationpage

\begin{dedicatory}
I dedicate this dissertation to myself and all my family, friends and
professors who gave me all necessary support to get here.
\end{dedicatory}

\acknowledgements
I would like to thank and dedicate this dissertation to the following people:
\ldots

\begin{epigraph}[Open My Eyes]{S.O.J.A.}
I open my eyes each morning I rise, to find a true\\
thought, know that it's real, I'm lucky to breathe,\\
I'm lucky to feel, I'm glad to wake up, I'm glad to be\\
here, with all of this world, and all of it's pain, all\\
of it's lies, and all of it's flipped down, I still\\
feel a sense of freedom, so glad I'm around,\\
\vspace{0.5cm}
It's my freedom, can't take it from me, i know it, it\\
won't change, but we need some understanding, I know\\
we'll be all right.
\end{epigraph}

\resumo
% Escreva seu resumo no arquivo resumo.tex
O gerenciamento eficiente de solicitações de mudança (SM) é fundamental para o
sucesso das atividades de manutenção e evolução de software. Entretanto, a
atribuição de SMs a desenvolvedores de software é um aspecto custoso desse
gerenciamento, pois demanda tempo e requer conhecimento apropriado do projeto de
software. Com o propósito de diminuir esse custo, várias pesquisas já propuseram
métodos de atribuição automática de SMs. Embora representem avanços na área,
existem vários fatores inerentes a atribuição de SMs que não são considerados
nessas pesquisas e são essenciais para a automação.

Como demonstrado nesse trabalho, a atribuição automática deve, por exemplo,
considerar a carga de trabalho, a experiência e o conhecimento dos
desenvolvedores, a prioridade e a severidade das SMs, a afinidade dos
desenvolvedores com os problemas descritos nas SMs, e até mesmo os
relacionamentos interpessoais. Para tornar esse cenário ainda mais complexo,
esses fatos podem variar de acordo com o projeto de software que está sendo
desenvolvido. Assim, uma solução para o problema de atribuição de SMs depende de
informações contextuais.

Assim, esse trabalho propõe, implementa e valida uma solução arquitetural
sensível ao contexto para atribuição automática de SMs. Dado o aspecto
contextual da solução, a arquitetura enfatiza a necessidade de considerar as
diversas fontes de informações presentes na organização, assim como a
necessidade de se desenvolver algorítimos que implementem diferentes estratégias
de atribuição. A proposta e implementação dessa solução é embasada em resultados
de duas pesquisas quantitativas: um estudo de mapeamento sistemático da
literatura, e uma pesquisa de questionário com desenvolvedores de software. Esse
último forneceu um conjunto de requisitos que a solução automatizada deve
satisfazer para que as estratégias de atribuição sejam atendidas, enquanto o
mapeamento da literatura identificou técnicas, algoritmos, e outros requisitos
necessários a automação.

A implementação da arquitetura segue uma estratégia de automação, também
elabo\-rada nesse trabalho, que possui dois componentes principais: um sistema
especialista baseado em regras (SEBR); e um modelo de recuperação de informação
(MRI) com técnicas de aprendizagem. Em nossa estratégia, esses dois componentes
são executados alternadamente em momentos diferentes a fim de atribuir uma SM
automaticamente. O SEBR processa regras simples e complexas, considerando
informações contextuais do projeto de software e da organização que o
desenvolve. O MRI é utilizado para fazer o casamento entre SMs e desenvolvedores
de acordo com o histórico de atribuições.

\begin{keywords}
Engenharia de Software, Manutenção e Evolução de Software, Gerenciamento de
Solicitações de Mudança, Atribuição Automática de Solicitações de Mudança
\end{keywords}

\abstract
% Write your abstract in a file called abstract.tex
Software maintenance and evolution are characterised by their huge cost and
slow speed of implementation. Yet they are inevitable activities -- almost all
software that is useful and successful stimulates user-generated requests for
change and improvements. \citeauthor{Sommerville2007} is
even more emphatic and says that software changes is a fact of life for large
software systems. In addition, a set of studies has stated along the years that
software maintenance and evolution is the most expensive phase of software
development, taking up to 90\% of the total costs.

All those characteristics from software maintenance lead the academia and
industry to constantly investigate new solutions to reduce costs in such
phase. In this context, Software Configuration Management (SCM) is a set of
activities and standards for managing and evolving software; SCM defines how
to record and process proposed system changes, how to relate these to system
components, among other procedures. For all these tasks it has been proposed
different tools, such as version control systems and bug trackers. However, some
issues may arise due to these tools usage, such as the dynamic assignment of a
developer to a bug report or the bug report duplication problem.

In this sense, this dissertation investigates the problem of bug report
duplication emerged by the use of bug trackers on software development
projects. The problem of bug report duplication is characterized by the
submission of two or more bug reports that describe the same software issue, and
the main consequence of this problem is the overhead of rework when managing
these bug reports.

\begin{keywords}
bug reports, bug trackers, bug report duplication, change request,
tool experiment, bug report duplication characterization study, bug report
search and analysis tool
\end{keywords}

% Summary (tables of contents)
\tableofcontents

% List of figures
\listoffigures

% List of tables
\listoftables

% List of acronyms
% Acronyms manual: http://linorg.usp.br/CTAN/macros/latex/contrib/acronym/acronym.pdf
\listofacronyms
\chapter*{List of Acronyms}
\addcontentsline{toc}{chapter}{List of Acronyms}
\begin{acronym}[C.E.S.A.R.]
  \acro{AJAX}{Asynchronous JavaScript and XML}
  \acro{BAST}{Bug Report Analysis and Search Tool}
  \acro{BTT}{Bug Report Tracker Tool}
  \acro{BRN}{Bug Report Network}
  \acro{CCB}{Change Control Board}
  \acro{C.E.S.A.R.}{Recife Center For Advanced Studies and Systems
  \acroextra{C.E.S.A.R. (\url{http://www.cesar.org.br}) is a CMMi level 3
  company with around 700 employees}}
  \acro{FR}{Functional Requirement}
  \acro{GQM}{Goal Question Metric}
  \acro{LOC}{Lines of Code}
  \acro{NFR}{Non-Functional Requirement}
  \acro{NLP}{Natural Language Processing}
  \acro{ORM}{Object-Relational Mapper}
  \acro{RiSE}{Reuse in Software Engineering Group \acroextra{\url{http://www.rise.com.br}}}
  \acro{SCM}{Software Configuration Management}
  \acro{SD}{Standard Deviation}
  \acro{TF-IDF}{Term Frequency-Inverse Document Frequency}
  \acro{UFPE}{Federal University of Pernambuco}
  \acro{VSM}{Vector Space Model}
  \acro{WAD}{Work as Design}
  \acro{XP}{eXtreme Programming}
\end{acronym}

\mainmatter

%% RiSE Latex Template - version 0.5
%%
%% RiSE's latex template for thesis and dissertations
%% http://risetemplate.sourceforge.net
%%
%% (c) 2012 Yguaratã Cerqueira Cavalcanti (yguarata@gmail.com)
%%          Vinicius Cardoso Garcia (vinicius.garcia@gmail.com)
%%
%% This document was initially based on UFPEThesis template, from Paulo Gustavo
%% S. Fonseca.
%%
%% ACKNOWLEDGEMENTS
%%
%% We would like to thanks the RiSE's researchers community, the 
%% students from Federal University of Pernambuco, and other users that have
%% been contributing to this projects with comments and patches.
%%
%% GENERAL INSTRUCTIONS
%%
%% We strongly recommend you to compile your documents using pdflatex command.
%% It is also recommend use the texlipse plugin for Eclipse to edit your documents.
%%
%% Options for \documentclass command:
%%         * Idiom
%%           pt   - Portguese (default)
%%           en   - English
%%
%%         * Text type
%%           bsc  - B.Sc. Thesis
%%           msc  - M.Sc. Thesis (default)
%%           qual - PHD qualification (not tested yet)
%%           prop - PHD proposal (not tested yet)
%%           phd  - PHD thesis
%%
%%         * Media
%%           scr  - to eletronic version (PDF) / see the users guide
%%
%%         * Pagination
%%           oneside - unique face press
%%           twoside - two faces press
%%
%%		   * Line spacing
%%           singlespacing  - the same as using \linespread{1}
%%           onehalfspacing - the same as using \linespread{1.3}
%%           doublespacing  - the same as using \linespread{1.6}
%%
%% Reference commands. Use the following commands to make references in your
%% text:
%%          \figref  -- for Figure reference
%%          \tabref  -- for Table reference
%%          \eqnref  -- for equation reference
%%          \chapref -- for chapter reference
%%          \secref  -- for section reference
%%          \appref  -- for appendix reference
%%          \axiref  -- for axiom reference
%%          \conjref -- for conjecture reference
%%          \defref  -- for definition reference
%%          \lemref  -- for lemma reference
%%          \theoref -- for theorem reference
%%          \corref  -- for corollary reference
%%          \propref -- for proprosition reference
%%          \pgref   -- for page reference
%%
%%          Example: See \chapref{chap:introduction}. It will produce 
%%                   'See Chapter 1', in case of English language.

\documentclass[en,twoside,onehalfspacing,msc]{risethesis}

\usepackage{natbib}
\usepackage{babel}
\usepackage{supertabular}
\usepackage{microtype}

%% Change the following pdf author attribute name to your name.
\usepackage[linkcolor=blue,citecolor=blue,urlcolor=blue,colorlinks,pdfpagelabels,pdftitle={rise-template-msc},pdfauthor={RiSE Latex Template}]{hyperref}

\address{RECIFE}

\universitypt{Universidade Federal de Pernambuco}
\universityen{Federal University of Pernambuco}

\departmentpt{Centro de Informática}
\departmenten{Center for Informatics}

\programpt{Pós-graduação em Ciência da Computação}
\programen{Graduate in Computer Science}

\majorfieldpt{Ciência da Computação}
\majorfielden{Computer Science}

\title{A Bug Report Analysis and Search Tool}
\date{JULY/2009}

\author{Yguaratã Cerqueira Cavalcanti}
\adviser{Silvio Romero de Lemos Meira}
\coadviser{Eduardo Santana de Almeida}

\begin{document}

\frontmatter
\frontpage
\presentationpage

\begin{dedicatory}
I dedicate this dissertation to myself and all my family, friends and
professors who gave me all necessary support to get here.
\end{dedicatory}

\acknowledgements
I would like to thank and dedicate this dissertation to the following people:
\ldots

\begin{epigraph}[Open My Eyes]{S.O.J.A.}
I open my eyes each morning I rise, to find a true\\
thought, know that it's real, I'm lucky to breathe,\\
I'm lucky to feel, I'm glad to wake up, I'm glad to be\\
here, with all of this world, and all of it's pain, all\\
of it's lies, and all of it's flipped down, I still\\
feel a sense of freedom, so glad I'm around,\\
\vspace{0.5cm}
It's my freedom, can't take it from me, i know it, it\\
won't change, but we need some understanding, I know\\
we'll be all right.
\end{epigraph}

\resumo
% Escreva seu resumo no arquivo resumo.tex
O gerenciamento eficiente de solicitações de mudança (SM) é fundamental para o
sucesso das atividades de manutenção e evolução de software. Entretanto, a
atribuição de SMs a desenvolvedores de software é um aspecto custoso desse
gerenciamento, pois demanda tempo e requer conhecimento apropriado do projeto de
software. Com o propósito de diminuir esse custo, várias pesquisas já propuseram
métodos de atribuição automática de SMs. Embora representem avanços na área,
existem vários fatores inerentes a atribuição de SMs que não são considerados
nessas pesquisas e são essenciais para a automação.

Como demonstrado nesse trabalho, a atribuição automática deve, por exemplo,
considerar a carga de trabalho, a experiência e o conhecimento dos
desenvolvedores, a prioridade e a severidade das SMs, a afinidade dos
desenvolvedores com os problemas descritos nas SMs, e até mesmo os
relacionamentos interpessoais. Para tornar esse cenário ainda mais complexo,
esses fatos podem variar de acordo com o projeto de software que está sendo
desenvolvido. Assim, uma solução para o problema de atribuição de SMs depende de
informações contextuais.

Assim, esse trabalho propõe, implementa e valida uma solução arquitetural
sensível ao contexto para atribuição automática de SMs. Dado o aspecto
contextual da solução, a arquitetura enfatiza a necessidade de considerar as
diversas fontes de informações presentes na organização, assim como a
necessidade de se desenvolver algorítimos que implementem diferentes estratégias
de atribuição. A proposta e implementação dessa solução é embasada em resultados
de duas pesquisas quantitativas: um estudo de mapeamento sistemático da
literatura, e uma pesquisa de questionário com desenvolvedores de software. Esse
último forneceu um conjunto de requisitos que a solução automatizada deve
satisfazer para que as estratégias de atribuição sejam atendidas, enquanto o
mapeamento da literatura identificou técnicas, algoritmos, e outros requisitos
necessários a automação.

A implementação da arquitetura segue uma estratégia de automação, também
elabo\-rada nesse trabalho, que possui dois componentes principais: um sistema
especialista baseado em regras (SEBR); e um modelo de recuperação de informação
(MRI) com técnicas de aprendizagem. Em nossa estratégia, esses dois componentes
são executados alternadamente em momentos diferentes a fim de atribuir uma SM
automaticamente. O SEBR processa regras simples e complexas, considerando
informações contextuais do projeto de software e da organização que o
desenvolve. O MRI é utilizado para fazer o casamento entre SMs e desenvolvedores
de acordo com o histórico de atribuições.

\begin{keywords}
Engenharia de Software, Manutenção e Evolução de Software, Gerenciamento de
Solicitações de Mudança, Atribuição Automática de Solicitações de Mudança
\end{keywords}

\abstract
% Write your abstract in a file called abstract.tex
Software maintenance and evolution are characterised by their huge cost and
slow speed of implementation. Yet they are inevitable activities -- almost all
software that is useful and successful stimulates user-generated requests for
change and improvements. \citeauthor{Sommerville2007} is
even more emphatic and says that software changes is a fact of life for large
software systems. In addition, a set of studies has stated along the years that
software maintenance and evolution is the most expensive phase of software
development, taking up to 90\% of the total costs.

All those characteristics from software maintenance lead the academia and
industry to constantly investigate new solutions to reduce costs in such
phase. In this context, Software Configuration Management (SCM) is a set of
activities and standards for managing and evolving software; SCM defines how
to record and process proposed system changes, how to relate these to system
components, among other procedures. For all these tasks it has been proposed
different tools, such as version control systems and bug trackers. However, some
issues may arise due to these tools usage, such as the dynamic assignment of a
developer to a bug report or the bug report duplication problem.

In this sense, this dissertation investigates the problem of bug report
duplication emerged by the use of bug trackers on software development
projects. The problem of bug report duplication is characterized by the
submission of two or more bug reports that describe the same software issue, and
the main consequence of this problem is the overhead of rework when managing
these bug reports.

\begin{keywords}
bug reports, bug trackers, bug report duplication, change request,
tool experiment, bug report duplication characterization study, bug report
search and analysis tool
\end{keywords}

% Summary (tables of contents)
\tableofcontents

% List of figures
\listoffigures

% List of tables
\listoftables

% List of acronyms
% Acronyms manual: http://linorg.usp.br/CTAN/macros/latex/contrib/acronym/acronym.pdf
\listofacronyms
\chapter*{List of Acronyms}
\addcontentsline{toc}{chapter}{List of Acronyms}
\begin{acronym}[C.E.S.A.R.]
  \acro{AJAX}{Asynchronous JavaScript and XML}
  \acro{BAST}{Bug Report Analysis and Search Tool}
  \acro{BTT}{Bug Report Tracker Tool}
  \acro{BRN}{Bug Report Network}
  \acro{CCB}{Change Control Board}
  \acro{C.E.S.A.R.}{Recife Center For Advanced Studies and Systems
  \acroextra{C.E.S.A.R. (\url{http://www.cesar.org.br}) is a CMMi level 3
  company with around 700 employees}}
  \acro{FR}{Functional Requirement}
  \acro{GQM}{Goal Question Metric}
  \acro{LOC}{Lines of Code}
  \acro{NFR}{Non-Functional Requirement}
  \acro{NLP}{Natural Language Processing}
  \acro{ORM}{Object-Relational Mapper}
  \acro{RiSE}{Reuse in Software Engineering Group \acroextra{\url{http://www.rise.com.br}}}
  \acro{SCM}{Software Configuration Management}
  \acro{SD}{Standard Deviation}
  \acro{TF-IDF}{Term Frequency-Inverse Document Frequency}
  \acro{UFPE}{Federal University of Pernambuco}
  \acro{VSM}{Vector Space Model}
  \acro{WAD}{Work as Design}
  \acro{XP}{eXtreme Programming}
\end{acronym}

\mainmatter

%% RiSE Latex Template - version 0.5
%%
%% RiSE's latex template for thesis and dissertations
%% http://risetemplate.sourceforge.net
%%
%% (c) 2012 Yguaratã Cerqueira Cavalcanti (yguarata@gmail.com)
%%          Vinicius Cardoso Garcia (vinicius.garcia@gmail.com)
%%
%% This document was initially based on UFPEThesis template, from Paulo Gustavo
%% S. Fonseca.
%%
%% ACKNOWLEDGEMENTS
%%
%% We would like to thanks the RiSE's researchers community, the 
%% students from Federal University of Pernambuco, and other users that have
%% been contributing to this projects with comments and patches.
%%
%% GENERAL INSTRUCTIONS
%%
%% We strongly recommend you to compile your documents using pdflatex command.
%% It is also recommend use the texlipse plugin for Eclipse to edit your documents.
%%
%% Options for \documentclass command:
%%         * Idiom
%%           pt   - Portguese (default)
%%           en   - English
%%
%%         * Text type
%%           bsc  - B.Sc. Thesis
%%           msc  - M.Sc. Thesis (default)
%%           qual - PHD qualification (not tested yet)
%%           prop - PHD proposal (not tested yet)
%%           phd  - PHD thesis
%%
%%         * Media
%%           scr  - to eletronic version (PDF) / see the users guide
%%
%%         * Pagination
%%           oneside - unique face press
%%           twoside - two faces press
%%
%%		   * Line spacing
%%           singlespacing  - the same as using \linespread{1}
%%           onehalfspacing - the same as using \linespread{1.3}
%%           doublespacing  - the same as using \linespread{1.6}
%%
%% Reference commands. Use the following commands to make references in your
%% text:
%%          \figref  -- for Figure reference
%%          \tabref  -- for Table reference
%%          \eqnref  -- for equation reference
%%          \chapref -- for chapter reference
%%          \secref  -- for section reference
%%          \appref  -- for appendix reference
%%          \axiref  -- for axiom reference
%%          \conjref -- for conjecture reference
%%          \defref  -- for definition reference
%%          \lemref  -- for lemma reference
%%          \theoref -- for theorem reference
%%          \corref  -- for corollary reference
%%          \propref -- for proprosition reference
%%          \pgref   -- for page reference
%%
%%          Example: See \chapref{chap:introduction}. It will produce 
%%                   'See Chapter 1', in case of English language.

\documentclass[en,twoside,onehalfspacing,msc]{risethesis}

\usepackage{natbib}
\usepackage{babel}
\usepackage{supertabular}
\usepackage{microtype}

%% Change the following pdf author attribute name to your name.
\usepackage[linkcolor=blue,citecolor=blue,urlcolor=blue,colorlinks,pdfpagelabels,pdftitle={rise-template-msc},pdfauthor={RiSE Latex Template}]{hyperref}

\address{RECIFE}

\universitypt{Universidade Federal de Pernambuco}
\universityen{Federal University of Pernambuco}

\departmentpt{Centro de Informática}
\departmenten{Center for Informatics}

\programpt{Pós-graduação em Ciência da Computação}
\programen{Graduate in Computer Science}

\majorfieldpt{Ciência da Computação}
\majorfielden{Computer Science}

\title{A Bug Report Analysis and Search Tool}
\date{JULY/2009}

\author{Yguaratã Cerqueira Cavalcanti}
\adviser{Silvio Romero de Lemos Meira}
\coadviser{Eduardo Santana de Almeida}

\begin{document}

\frontmatter
\frontpage
\presentationpage

\begin{dedicatory}
I dedicate this dissertation to myself and all my family, friends and
professors who gave me all necessary support to get here.
\end{dedicatory}

\acknowledgements
I would like to thank and dedicate this dissertation to the following people:
\ldots

\begin{epigraph}[Open My Eyes]{S.O.J.A.}
I open my eyes each morning I rise, to find a true\\
thought, know that it's real, I'm lucky to breathe,\\
I'm lucky to feel, I'm glad to wake up, I'm glad to be\\
here, with all of this world, and all of it's pain, all\\
of it's lies, and all of it's flipped down, I still\\
feel a sense of freedom, so glad I'm around,\\
\vspace{0.5cm}
It's my freedom, can't take it from me, i know it, it\\
won't change, but we need some understanding, I know\\
we'll be all right.
\end{epigraph}

\resumo
% Escreva seu resumo no arquivo resumo.tex
\input{resumo}

\abstract
% Write your abstract in a file called abstract.tex
\input{abstract}

% Summary (tables of contents)
\tableofcontents

% List of figures
\listoffigures

% List of tables
\listoftables

% List of acronyms
% Acronyms manual: http://linorg.usp.br/CTAN/macros/latex/contrib/acronym/acronym.pdf
\listofacronyms
\input{acronyms}

\mainmatter

\include{chapters/introduction/main}

\bibliographystyle{natbib}
\addcontentsline{toc}{chapter}{\bibliographytocname}
\bibliography{references}

% Appendix
\clearpage
\addappheadtotoc
\appendix
\appendixpage
\include{appendix/experiment-instruments}

\end{document}


\bibliographystyle{natbib}
\addcontentsline{toc}{chapter}{\bibliographytocname}
\bibliography{references}

% Appendix
\clearpage
\addappheadtotoc
\appendix
\appendixpage
\chapter{Experiment Instruments}
\label{ap:measurement-instruments}


\def \tick{
$[$\hspace{0.3cm}$]$
}

\def \twooption#1#2{
\tick #1.  \tick #2.
}

\def \threeoption#1#2#3{
\tick #1.\newline
\tick #2.\newline
\tick #3.
}

\def \fouroption#1#2#3#4{
\tick #1.\newline
\tick #2.\newline
\tick #3.\newline
\tick #4.
}

\def \fiveoption#1#2#3#4#5{
\tick #1.\newline
\tick #2.\newline
\tick #3.\newline
\tick #4.\newline
\tick #5.
}
\def \datefield{
%date fied used in time sheets
    /\hspace{0.4cm}/
}

\def \rcolor{
%table row color
    \rowcolor[gray]{0.9}
}

\def \hcolor{
    \rowcolor[gray]{0.7}
}

\section{Time sheet}
\label{ap:sec:time-sheet}

\begin{table*}[h]
  \centering
  \begin{tabular}{|c|c|c|c|c|p{5cm}|}
    \hline
    \hcolor ID & $Start~date$ & $Start~time$ & $End~date$ & $End~time$ & 
        $Is~it~a~duplicate?$\\
    \hline
       1 & \datefield  & : & \datefield & : & \twooption{Yes}{No} ID:\\
    \hline
       \rcolor 2 & \datefield  &  : & \datefield & : & \twooption{Yes}{No} ID:
       \\
    \hline
       3 & \datefield  &  : & \datefield & : & \twooption{Yes}{No} ID:\\
    \hline
       \rcolor 4 & \datefield  &  : & \datefield & : & \twooption{Yes}{No} ID:\\
    \hline
       5 & \datefield  &  : & \datefield & : & \twooption{Yes}{No} ID:\\
    \hline
       \rcolor 6 & \datefield  &  : & \datefield & : & \twooption{Yes}{No} ID:\\
    \hline
       7 & \datefield  &  : & \datefield & : & \twooption{Yes}{No} ID:\\
    \hline
       \rcolor 8 & \datefield  &  : & \datefield & : & \twooption{Yes}{No} ID:\\
    \hline
       9 & \datefield  &  : & \datefield & : & \twooption{Yes}{No} ID:\\
    \hline
       \rcolor 10 & \datefield  &  : & \datefield & : & \twooption{Yes}{No} ID:\\
    \hline
        \multicolumn{6}{|c|}{\ldots}\\
    \hline
       23 & \datefield  &  : & \datefield & : & \twooption{Yes}{No} ID:\\
    \hline
       \rcolor 24 & \datefield  &  : & \datefield & : & \twooption{Yes}{No} ID:\\
    \hline
       25 & \datefield  &  : & \datefield & : & \twooption{Yes}{No} ID:\\
    \hline
       \rcolor 26 & \datefield  &  : & \datefield & : & \twooption{Yes}{No} ID:\\
    \hline
       27 & \datefield  &  : & \datefield & : & \twooption{Yes}{No} ID:\\
    \hline
       \rcolor 28 & \datefield  &  : & \datefield & : & \twooption{Yes}{No} ID:\\
    \hline
       29 & \datefield  &  : & \datefield & : & \twooption{Yes}{No} ID:\\
    \hline
       \rcolor 30 & \datefield  &  : & \datefield & : & \twooption{Yes}{No} ID:\\
    \hline
       31 & \datefield  &  : & \datefield & : & \twooption{Yes}{No} ID:\\
    \hline
       \rcolor 32 & \datefield  &  : & \datefield & : & \twooption{Yes}{No} ID:\\
    \hline
  \end{tabular}
\caption{Time sheet used in the study.
%The subjects use this time
%sheet to note the start and end timestamps of a bug-report analysis, and mark
%if has been found a duplicate for the bug-report under analysis.
}
\end{table*}

\newpage
\section{Questionnaire for Subjects Profile}
\label{ap:sec:profile}

\begin{table}[h]
\centering
\resizebox{15cm}{!}{
\begin{tabular}{|p{\columnwidth}|}
    \hline
    \hcolor \textbf{\textsf{Questionnaire for Subjects Profile}}\\
    \hline
	    \textbf{How many years since graduation?}\\\\
	    \tick years.\\\\
    \hline
        \textbf{How many projects do you have participated according to the
        following categories?}\\ \\
        \tick Low complexity.\\
        \tick Medium complexity.\\
        \tick High complexity.\\\\
    \hline
        \textbf{What were the roles that you played in the projects cited before
        (developer, configuration manager, tester\ldots)?}
        \\
        \\
    \hline
        \textbf{How do you define your experience with bug-trackers?}\\\\
        \tick I never used them before.\\
        \tick I used them in every project i participated.\\
        I used them in \tick projects.\\\\
    \hline
        \textbf{Do you have used any of the following bug-trackers?}\\\\
        \tick Bugzilla. In: \tick industry  \tick academia\\
        \tick Trac. In: \tick industry  \tick academia\\
        \tick Mantis. In: \tick industry  \tick academia\\
        \tick Jyra. In: \tick industry  \tick academia\\
        \tick BSD Bug-tracker. In: \tick industry  \tick academia\\
        \tick Other: \\\\
    \hline
        \textbf{Have you performed any analysis of Firefox bug-reports before?}
        \\\\\twooption{Yes}{No}\\\\
    \hline
\end{tabular}
}
\caption{Questionnaire for bug-report submitters.}
\end{table}

\newpage
\section{Form for Qualitative Analysis}
\label{ap:sec:feedback}

\begin{table}[h]
\centering
\resizebox{14cm}{!}{
\begin{tabular}{|p{\columnwidth}|}
    \hline
    \hcolor \textbf{\textsf{Questionnaire for Qualitative Analysis}}\\
    \hline
        \textbf{Did you use any of the search filters provided by BAST?}
        \\
        \twooption{Yes}{No}
        \\
    \hline
        \textbf{Is there any search filter you think it must be present in BAST?}
        \\
        \twooption{Yes}{No}
        Cite them:
        \\
    \hline
        \textbf{Did you have any problem with the search filters usage?}
        \\
        \twooption{Yes}{No}
        Cite them:
        \\
    \hline
        \textbf{Did you use the ordering features of BAST?}
        \\
        \twooption{Yes}{No}
        \\
    \hline
        \textbf{Did you have any problem with ordering features?}
        \\
        \twooption{Yes}{No}
        Cite them:
        \\
    \hline
        \textbf{Do you think there is any other important information that
        must be present in the list of search results?}
        \\
        \twooption{Yes}{No}
        Cite them:
        \\
    \hline
        \textbf{Did you have any problem to visualize the details from some
        bug-report?} \\
        \twooption{Yes}{No}
        Cite them:
        \\
    \hline
        \textbf{Do you believe the way bug-reports details are presented was
        helpful to perform the analysis?}
        \\
        \twooption{Yes}{No}
        \\
    \hline 
        \textbf{Was the recommendation of related bug-reports, presented  in
        the bug-report details, useful for the analysis?}
        \\
        \twooption{Yes}{No}
        \\
    \hline
        \textbf{Is there any other information concerning bug-reports details
        you believe it should be present or emphasized?}
        \\
        \twooption{Yes}{No}
        Cite them:
        \\\\
    \hline
        \textbf{Did you use the help provided by BAST?}
        \\
        \twooption{Yes}{No}
        \\
    \hline
        \textbf{Did you found any other problem/enhacement/defect that was not
        mentioned before? Cite them.}
        \\
        \\
        \\
    \hline
        \textbf{Please, write down any suggestion you think might would be useful.}
        \\
        \\
        \\
    \hline
\end{tabular}
}
\caption{Questionnaire for qualitative analysis.}
\end{table}

\end{document}


\bibliographystyle{natbib}
\addcontentsline{toc}{chapter}{\bibliographytocname}
\bibliography{references}

% Appendix
\clearpage
\addappheadtotoc
\appendix
\appendixpage
\chapter{Experiment Instruments}
\label{ap:measurement-instruments}


\def \tick{
$[$\hspace{0.3cm}$]$
}

\def \twooption#1#2{
\tick #1.  \tick #2.
}

\def \threeoption#1#2#3{
\tick #1.\newline
\tick #2.\newline
\tick #3.
}

\def \fouroption#1#2#3#4{
\tick #1.\newline
\tick #2.\newline
\tick #3.\newline
\tick #4.
}

\def \fiveoption#1#2#3#4#5{
\tick #1.\newline
\tick #2.\newline
\tick #3.\newline
\tick #4.\newline
\tick #5.
}
\def \datefield{
%date fied used in time sheets
    /\hspace{0.4cm}/
}

\def \rcolor{
%table row color
    \rowcolor[gray]{0.9}
}

\def \hcolor{
    \rowcolor[gray]{0.7}
}

\section{Time sheet}
\label{ap:sec:time-sheet}

\begin{table*}[h]
  \centering
  \begin{tabular}{|c|c|c|c|c|p{5cm}|}
    \hline
    \hcolor ID & $Start~date$ & $Start~time$ & $End~date$ & $End~time$ & 
        $Is~it~a~duplicate?$\\
    \hline
       1 & \datefield  & : & \datefield & : & \twooption{Yes}{No} ID:\\
    \hline
       \rcolor 2 & \datefield  &  : & \datefield & : & \twooption{Yes}{No} ID:
       \\
    \hline
       3 & \datefield  &  : & \datefield & : & \twooption{Yes}{No} ID:\\
    \hline
       \rcolor 4 & \datefield  &  : & \datefield & : & \twooption{Yes}{No} ID:\\
    \hline
       5 & \datefield  &  : & \datefield & : & \twooption{Yes}{No} ID:\\
    \hline
       \rcolor 6 & \datefield  &  : & \datefield & : & \twooption{Yes}{No} ID:\\
    \hline
       7 & \datefield  &  : & \datefield & : & \twooption{Yes}{No} ID:\\
    \hline
       \rcolor 8 & \datefield  &  : & \datefield & : & \twooption{Yes}{No} ID:\\
    \hline
       9 & \datefield  &  : & \datefield & : & \twooption{Yes}{No} ID:\\
    \hline
       \rcolor 10 & \datefield  &  : & \datefield & : & \twooption{Yes}{No} ID:\\
    \hline
        \multicolumn{6}{|c|}{\ldots}\\
    \hline
       23 & \datefield  &  : & \datefield & : & \twooption{Yes}{No} ID:\\
    \hline
       \rcolor 24 & \datefield  &  : & \datefield & : & \twooption{Yes}{No} ID:\\
    \hline
       25 & \datefield  &  : & \datefield & : & \twooption{Yes}{No} ID:\\
    \hline
       \rcolor 26 & \datefield  &  : & \datefield & : & \twooption{Yes}{No} ID:\\
    \hline
       27 & \datefield  &  : & \datefield & : & \twooption{Yes}{No} ID:\\
    \hline
       \rcolor 28 & \datefield  &  : & \datefield & : & \twooption{Yes}{No} ID:\\
    \hline
       29 & \datefield  &  : & \datefield & : & \twooption{Yes}{No} ID:\\
    \hline
       \rcolor 30 & \datefield  &  : & \datefield & : & \twooption{Yes}{No} ID:\\
    \hline
       31 & \datefield  &  : & \datefield & : & \twooption{Yes}{No} ID:\\
    \hline
       \rcolor 32 & \datefield  &  : & \datefield & : & \twooption{Yes}{No} ID:\\
    \hline
  \end{tabular}
\caption{Time sheet used in the study.
%The subjects use this time
%sheet to note the start and end timestamps of a bug-report analysis, and mark
%if has been found a duplicate for the bug-report under analysis.
}
\end{table*}

\newpage
\section{Questionnaire for Subjects Profile}
\label{ap:sec:profile}

\begin{table}[h]
\centering
\resizebox{15cm}{!}{
\begin{tabular}{|p{\columnwidth}|}
    \hline
    \hcolor \textbf{\textsf{Questionnaire for Subjects Profile}}\\
    \hline
	    \textbf{How many years since graduation?}\\\\
	    \tick years.\\\\
    \hline
        \textbf{How many projects do you have participated according to the
        following categories?}\\ \\
        \tick Low complexity.\\
        \tick Medium complexity.\\
        \tick High complexity.\\\\
    \hline
        \textbf{What were the roles that you played in the projects cited before
        (developer, configuration manager, tester\ldots)?}
        \\
        \\
    \hline
        \textbf{How do you define your experience with bug-trackers?}\\\\
        \tick I never used them before.\\
        \tick I used them in every project i participated.\\
        I used them in \tick projects.\\\\
    \hline
        \textbf{Do you have used any of the following bug-trackers?}\\\\
        \tick Bugzilla. In: \tick industry  \tick academia\\
        \tick Trac. In: \tick industry  \tick academia\\
        \tick Mantis. In: \tick industry  \tick academia\\
        \tick Jyra. In: \tick industry  \tick academia\\
        \tick BSD Bug-tracker. In: \tick industry  \tick academia\\
        \tick Other: \\\\
    \hline
        \textbf{Have you performed any analysis of Firefox bug-reports before?}
        \\\\\twooption{Yes}{No}\\\\
    \hline
\end{tabular}
}
\caption{Questionnaire for bug-report submitters.}
\end{table}

\newpage
\section{Form for Qualitative Analysis}
\label{ap:sec:feedback}

\begin{table}[h]
\centering
\resizebox{14cm}{!}{
\begin{tabular}{|p{\columnwidth}|}
    \hline
    \hcolor \textbf{\textsf{Questionnaire for Qualitative Analysis}}\\
    \hline
        \textbf{Did you use any of the search filters provided by BAST?}
        \\
        \twooption{Yes}{No}
        \\
    \hline
        \textbf{Is there any search filter you think it must be present in BAST?}
        \\
        \twooption{Yes}{No}
        Cite them:
        \\
    \hline
        \textbf{Did you have any problem with the search filters usage?}
        \\
        \twooption{Yes}{No}
        Cite them:
        \\
    \hline
        \textbf{Did you use the ordering features of BAST?}
        \\
        \twooption{Yes}{No}
        \\
    \hline
        \textbf{Did you have any problem with ordering features?}
        \\
        \twooption{Yes}{No}
        Cite them:
        \\
    \hline
        \textbf{Do you think there is any other important information that
        must be present in the list of search results?}
        \\
        \twooption{Yes}{No}
        Cite them:
        \\
    \hline
        \textbf{Did you have any problem to visualize the details from some
        bug-report?} \\
        \twooption{Yes}{No}
        Cite them:
        \\
    \hline
        \textbf{Do you believe the way bug-reports details are presented was
        helpful to perform the analysis?}
        \\
        \twooption{Yes}{No}
        \\
    \hline 
        \textbf{Was the recommendation of related bug-reports, presented  in
        the bug-report details, useful for the analysis?}
        \\
        \twooption{Yes}{No}
        \\
    \hline
        \textbf{Is there any other information concerning bug-reports details
        you believe it should be present or emphasized?}
        \\
        \twooption{Yes}{No}
        Cite them:
        \\\\
    \hline
        \textbf{Did you use the help provided by BAST?}
        \\
        \twooption{Yes}{No}
        \\
    \hline
        \textbf{Did you found any other problem/enhacement/defect that was not
        mentioned before? Cite them.}
        \\
        \\
        \\
    \hline
        \textbf{Please, write down any suggestion you think might would be useful.}
        \\
        \\
        \\
    \hline
\end{tabular}
}
\caption{Questionnaire for qualitative analysis.}
\end{table}

\end{document}


\bibliographystyle{natbib}
\addcontentsline{toc}{chapter}{\bibliographytocname}
\bibliography{references}

% Appendix
\clearpage
\addappheadtotoc
\appendix
\appendixpage
\chapter{Experiment Instruments}
\label{ap:measurement-instruments}


\def \tick{
$[$\hspace{0.3cm}$]$
}

\def \twooption#1#2{
\tick #1.  \tick #2.
}

\def \threeoption#1#2#3{
\tick #1.\newline
\tick #2.\newline
\tick #3.
}

\def \fouroption#1#2#3#4{
\tick #1.\newline
\tick #2.\newline
\tick #3.\newline
\tick #4.
}

\def \fiveoption#1#2#3#4#5{
\tick #1.\newline
\tick #2.\newline
\tick #3.\newline
\tick #4.\newline
\tick #5.
}
\def \datefield{
%date fied used in time sheets
    /\hspace{0.4cm}/
}

\def \rcolor{
%table row color
    \rowcolor[gray]{0.9}
}

\def \hcolor{
    \rowcolor[gray]{0.7}
}

\section{Time sheet}
\label{ap:sec:time-sheet}

\begin{table*}[h]
  \centering
  \begin{tabular}{|c|c|c|c|c|p{5cm}|}
    \hline
    \hcolor ID & $Start~date$ & $Start~time$ & $End~date$ & $End~time$ & 
        $Is~it~a~duplicate?$\\
    \hline
       1 & \datefield  & : & \datefield & : & \twooption{Yes}{No} ID:\\
    \hline
       \rcolor 2 & \datefield  &  : & \datefield & : & \twooption{Yes}{No} ID:
       \\
    \hline
       3 & \datefield  &  : & \datefield & : & \twooption{Yes}{No} ID:\\
    \hline
       \rcolor 4 & \datefield  &  : & \datefield & : & \twooption{Yes}{No} ID:\\
    \hline
       5 & \datefield  &  : & \datefield & : & \twooption{Yes}{No} ID:\\
    \hline
       \rcolor 6 & \datefield  &  : & \datefield & : & \twooption{Yes}{No} ID:\\
    \hline
       7 & \datefield  &  : & \datefield & : & \twooption{Yes}{No} ID:\\
    \hline
       \rcolor 8 & \datefield  &  : & \datefield & : & \twooption{Yes}{No} ID:\\
    \hline
       9 & \datefield  &  : & \datefield & : & \twooption{Yes}{No} ID:\\
    \hline
       \rcolor 10 & \datefield  &  : & \datefield & : & \twooption{Yes}{No} ID:\\
    \hline
        \multicolumn{6}{|c|}{\ldots}\\
    \hline
       23 & \datefield  &  : & \datefield & : & \twooption{Yes}{No} ID:\\
    \hline
       \rcolor 24 & \datefield  &  : & \datefield & : & \twooption{Yes}{No} ID:\\
    \hline
       25 & \datefield  &  : & \datefield & : & \twooption{Yes}{No} ID:\\
    \hline
       \rcolor 26 & \datefield  &  : & \datefield & : & \twooption{Yes}{No} ID:\\
    \hline
       27 & \datefield  &  : & \datefield & : & \twooption{Yes}{No} ID:\\
    \hline
       \rcolor 28 & \datefield  &  : & \datefield & : & \twooption{Yes}{No} ID:\\
    \hline
       29 & \datefield  &  : & \datefield & : & \twooption{Yes}{No} ID:\\
    \hline
       \rcolor 30 & \datefield  &  : & \datefield & : & \twooption{Yes}{No} ID:\\
    \hline
       31 & \datefield  &  : & \datefield & : & \twooption{Yes}{No} ID:\\
    \hline
       \rcolor 32 & \datefield  &  : & \datefield & : & \twooption{Yes}{No} ID:\\
    \hline
  \end{tabular}
\caption{Time sheet used in the study.
%The subjects use this time
%sheet to note the start and end timestamps of a bug-report analysis, and mark
%if has been found a duplicate for the bug-report under analysis.
}
\end{table*}

\newpage
\section{Questionnaire for Subjects Profile}
\label{ap:sec:profile}

\begin{table}[h]
\centering
\resizebox{15cm}{!}{
\begin{tabular}{|p{\columnwidth}|}
    \hline
    \hcolor \textbf{\textsf{Questionnaire for Subjects Profile}}\\
    \hline
	    \textbf{How many years since graduation?}\\\\
	    \tick years.\\\\
    \hline
        \textbf{How many projects do you have participated according to the
        following categories?}\\ \\
        \tick Low complexity.\\
        \tick Medium complexity.\\
        \tick High complexity.\\\\
    \hline
        \textbf{What were the roles that you played in the projects cited before
        (developer, configuration manager, tester\ldots)?}
        \\
        \\
    \hline
        \textbf{How do you define your experience with bug-trackers?}\\\\
        \tick I never used them before.\\
        \tick I used them in every project i participated.\\
        I used them in \tick projects.\\\\
    \hline
        \textbf{Do you have used any of the following bug-trackers?}\\\\
        \tick Bugzilla. In: \tick industry  \tick academia\\
        \tick Trac. In: \tick industry  \tick academia\\
        \tick Mantis. In: \tick industry  \tick academia\\
        \tick Jyra. In: \tick industry  \tick academia\\
        \tick BSD Bug-tracker. In: \tick industry  \tick academia\\
        \tick Other: \\\\
    \hline
        \textbf{Have you performed any analysis of Firefox bug-reports before?}
        \\\\\twooption{Yes}{No}\\\\
    \hline
\end{tabular}
}
\caption{Questionnaire for bug-report submitters.}
\end{table}

\newpage
\section{Form for Qualitative Analysis}
\label{ap:sec:feedback}

\begin{table}[h]
\centering
\resizebox{14cm}{!}{
\begin{tabular}{|p{\columnwidth}|}
    \hline
    \hcolor \textbf{\textsf{Questionnaire for Qualitative Analysis}}\\
    \hline
        \textbf{Did you use any of the search filters provided by BAST?}
        \\
        \twooption{Yes}{No}
        \\
    \hline
        \textbf{Is there any search filter you think it must be present in BAST?}
        \\
        \twooption{Yes}{No}
        Cite them:
        \\
    \hline
        \textbf{Did you have any problem with the search filters usage?}
        \\
        \twooption{Yes}{No}
        Cite them:
        \\
    \hline
        \textbf{Did you use the ordering features of BAST?}
        \\
        \twooption{Yes}{No}
        \\
    \hline
        \textbf{Did you have any problem with ordering features?}
        \\
        \twooption{Yes}{No}
        Cite them:
        \\
    \hline
        \textbf{Do you think there is any other important information that
        must be present in the list of search results?}
        \\
        \twooption{Yes}{No}
        Cite them:
        \\
    \hline
        \textbf{Did you have any problem to visualize the details from some
        bug-report?} \\
        \twooption{Yes}{No}
        Cite them:
        \\
    \hline
        \textbf{Do you believe the way bug-reports details are presented was
        helpful to perform the analysis?}
        \\
        \twooption{Yes}{No}
        \\
    \hline 
        \textbf{Was the recommendation of related bug-reports, presented  in
        the bug-report details, useful for the analysis?}
        \\
        \twooption{Yes}{No}
        \\
    \hline
        \textbf{Is there any other information concerning bug-reports details
        you believe it should be present or emphasized?}
        \\
        \twooption{Yes}{No}
        Cite them:
        \\\\
    \hline
        \textbf{Did you use the help provided by BAST?}
        \\
        \twooption{Yes}{No}
        \\
    \hline
        \textbf{Did you found any other problem/enhacement/defect that was not
        mentioned before? Cite them.}
        \\
        \\
        \\
    \hline
        \textbf{Please, write down any suggestion you think might would be useful.}
        \\
        \\
        \\
    \hline
\end{tabular}
}
\caption{Questionnaire for qualitative analysis.}
\end{table}

\end{document}
