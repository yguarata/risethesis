%% RiSE Latex Template - version 0.5
%%
%% RiSE's latex template for thesis and dissertations
%% http://risetemplate.sourceforge.net
%%
%% (c) 2012 Yguaratã Cerqueira Cavalcanti (yguarata@gmail.com)
%%          Vinicius Cardoso Garcia (vinicius.garcia@gmail.com)
%%
%% This document was initially based on UFPEThesis template, from Paulo Gustavo
%% S. Fonseca.
%%
%% ACKNOWLEDGEMENTS
%%
%% We would like to thanks the RiSE's researchers community, the 
%% students from Federal University of Pernambuco, and other users that have
%% been contributing to this projects with comments and patches.
%%
%% GENERAL INSTRUCTIONS
%%
%% We strongly recommend you to compile your documents using pdflatex command.
%% It is also recommend use the texlipse plugin for Eclipse to edit your documents.
%%
%% Options for \documentclass command:
%%         * Idiom
%%           pt   - Portguese (default)
%%           en   - English
%%
%%         * Text type
%%           bsc  - B.Sc. Thesis
%%           msc  - M.Sc. Thesis (default)
%%           qual - PHD qualification (not tested yet)
%%           prop - PHD proposal (not tested yet)
%%           phd  - PHD thesis
%%
%%         * Media
%%           scr  - to eletronic version (PDF) / see the users guide
%%
%%         * Pagination
%%           oneside - unique face press
%%           twoside - two faces press
%%
%%		   * Line spacing
%%           singlespacing  - the same as using \linespread{1}
%%           onehalfspacing - the same as using \linespread{1.3}
%%           doublespacing  - the same as using \linespread{1.6}
%%
%% Reference commands. Use the following commands to make references in your
%% text:
%%          \figref  -- for Figure reference
%%          \tabref  -- for Table reference
%%          \eqnref  -- for equation reference
%%          \chapref -- for chapter reference
%%          \secref  -- for section reference
%%          \appref  -- for appendix reference
%%          \axiref  -- for axiom reference
%%          \conjref -- for conjecture reference
%%          \defref  -- for definition reference
%%          \lemref  -- for lemma reference
%%          \theoref -- for theorem reference
%%          \corref  -- for corollary reference
%%          \propref -- for proprosition reference
%%          \pgref   -- for page reference
%%
%%          Example: See \chapref{chap:introduction}. It will produce 
%%                   'See Chapter 1', in case of English language.

\documentclass[en,twoside,onehalfspacing,msc]{risethesis}

\usepackage{natbib}
\usepackage{babel}
\usepackage{supertabular}
\usepackage{microtype}

%% Change the following pdf author attribute name to your name.
\usepackage[linkcolor=blue,citecolor=blue,urlcolor=blue,colorlinks,pdfpagelabels,pdftitle={rise-template-msc},pdfauthor={RiSE Latex Template}]{hyperref}

\address{RECIFE}

\universitypt{Universidade Federal de Pernambuco}
\universityen{Federal University of Pernambuco}

\departmentpt{Centro de Informática}
\departmenten{Center for Informatics}

\programpt{Pós-graduação em Ciência da Computação}
\programen{Graduate in Computer Science}

\majorfieldpt{Ciência da Computação}
\majorfielden{Computer Science}

\title{A Bug Report Analysis and Search Tool}
\date{JULY/2009}

\author{Yguaratã Cerqueira Cavalcanti}
\adviser{Silvio Romero de Lemos Meira}
\coadviser{Eduardo Santana de Almeida}

\begin{document}

\frontmatter
\frontpage
\presentationpage

\begin{dedicatory}
I dedicate this dissertation to myself and all my family, friends and
professors who gave me all necessary support to get here.
\end{dedicatory}

\acknowledgements
I would like to thank and dedicate this dissertation to the following people:
\ldots

\begin{epigraph}[Open My Eyes]{S.O.J.A.}
I open my eyes each morning I rise, to find a true\\
thought, know that it's real, I'm lucky to breathe,\\
I'm lucky to feel, I'm glad to wake up, I'm glad to be\\
here, with all of this world, and all of it's pain, all\\
of it's lies, and all of it's flipped down, I still\\
feel a sense of freedom, so glad I'm around,\\
\vspace{0.5cm}
It's my freedom, can't take it from me, i know it, it\\
won't change, but we need some understanding, I know\\
we'll be all right.
\end{epigraph}

\resumo
% Escreva seu resumo no arquivo resumo.tex
Manuten��o e evolu��o de \emph{software} s�o atividades caracterizadas pelo
seu enorme custo e baixa velocidade de execu��o. N�o obstante, elas s�o
atividades inevit�veis para garantir a qualidade do \emph{software} -- quase todo
\emph{software} bem sucedido estimula os usu�rios a fazer pedidos de mudan�as
e melhorias. Sommerville � ainda mais enf�tico e diz que mudan�as em projetos de
\emph{software} s�o um fato. Al�m disso, diferentes estudos t�m afirmado ao
longo dos anos que as atividades de manuten��o e evolu��o de \emph{software} s�o
as mais caras do ciclo de desenvolvimento, sendo respons�vel por cerca de
at� 90\% dos custos.

Todas essas peculiaridades da fase de manuten��o e evolu��o de \emph{software}
leva o mundo acad�mico e industrial a investigar constantemente novas
solu��es para reduzir os custos dessas atividades. Neste contexto, Ger�ncia de
Configura��o de Software (GCS) � um conjunto de atividades e normas para a
gest�o da evolu��o e manuten��o de \emph{software}; GCS define como s�o
registradas e processadas todas as modifica��es, o impacto das mesmas em todo
o sistema, dentre outros procedimentos. Para todas estas tarefas de GCM
existem diferentes ferramentas de aux�lio, tais como sistemas de controle de
vers�o e \emph{bug trackers}. No entanto, alguns problemas podem surgir devido ao uso
das mesmas, como por exemplo o problema de atribui��o autom�tica de respons�vel
por um \emph{bug report} e o problema de duplica��o de \emph{bug reports}.

Neste sentido, esta disserta��o investiga o problema de duplica��o de
\emph{bug reports} resultante da utiliza��o de \emph{bug trackers} em projetos
de desenvolvimento de \emph{software}. Tal problema � caracterizado pela
submiss�o de dois ou mais \emph{bug reports} que descrevem o mesmo problema
referente a um \emph{software}, tendo como principais conseq��ncias a
sobrecarga de trabalho na busca e an�lise de \emph{bug reports}, e o mal
aproveitamento do tempo destinado a essa atividade.

\begin{keywords}
relatos de bug, gerenciadores de relatos de bug, relatos de bug duplicados,
requisi��o de mudan�a, experimento, estudo de caracteriza��o, ferramenta, busca
\end{keywords}

\abstract
% Write your abstract in a file called abstract.tex
The efficient management of \acp{cr} is fundamental for successful software
maintenance; however the assignment of \acp{cr} to developers is an expensive
aspects in this regard, due to the time and expertise demanded. To overcome
this, researchers have proposed automated approaches for \ac{cr} assignment.
Although these proposals present advances to this topic, they do not consider
many factors inherent to the assignments. Indeed, different complex factors may
have influence on \ac{cr} assignment, and most of them vary from one
organization to another. For instance, developers' workload, \acp{cr} severity,
interpersonal relationships, or developers know-how must be considered in the
assignments. Actually, as we demonstrate in this work, \ac{cr} assignment is a
complex activity and automated approaches cannot rely on simplistic solutions.
Ideally, it is necessary to consider and reason over contextual information in
order to provide an effective automation.

In this regarding, this work proposes, implements, and validates a context-aware
architecture to automate \ac{cr} assignment. The architecture emphasizes the
need for considering the different information available at the organization to
provide a more context-aware solution to automated \ac{cr} assignment. The
development of such architecture is supported by evidence synthesized from two
empirical studies: a survey with practitioners and a systematic mapping study.
The survey provided us with a set of requirements that automated approaches
should satisfy. In the mapping study, in turn, we figured out how
state-of-the-art approaches are implemented in regarding to these requirements,
concluding that many of them are not satisfied. In addition, new requirements
were identified in this mapping study.

For the implementation of the proposed architecture, we developed a strategy to
automate \ac{cr} assignments which is based on two main components: a \acf{rbes}
and an \acf{ir} model. The strategy coordinately applies these two components in
different steps to find the potential developer to a \ac{cr}. The \ac{rbes}
takes care of the simple and complex rules necessary to consider contextual
information in the assignments, e.g., to prevent assigning a \ac{cr} to a busy
or unavailable developer. Since these rules vary from one organization/project
to another, the \ac{rbes} facilitates their modification for different contexts.
On the other hand, the \ac{ir} model is useful to make use of the historical
information of \ac{cr} assignments to match \acp{cr} and developers.

\begin{keywords}
Software Engineering, Software Maintenance and Evolution, Change Request
Management, Automatic Change Request Assignment
\end{keywords}

% Summary (tables of contents)
\tableofcontents

% List of figures
\listoffigures

% List of tables
\listoftables

% List of acronyms
% Acronyms manual: http://linorg.usp.br/CTAN/macros/latex/contrib/acronym/acronym.pdf
\listofacronyms
\begin{acronym}[ACRONYM] 
% Change the word ACRONYM above to change the acronym column width.
% The column width is equals to the width of the word that you put.
% Read the manual about acronym package for more examples:
%   http://linorg.usp.br/CTAN/macros/latex/contrib/acronym/acronym.pdf
  \acro{AJAX}{Asynchronous JavaScript and XML}
  \acro{BAST}{Bug Report Analysis and Search Tool}
  \acro{BTT}{Bug Report Tracker Tool}
  \acro{BRN}{Bug Report Network}
  \acro{CCB}{Change Control Board}
\end{acronym}

% List of listings
\lstlistoflistings

\mainmatter


%% RiSE's latex template for thesis and dissertations
%% Version 0.2
%% (c) 2008 Yguarat� Cerqueira Cavalcanti (yguarata@gmail.com)
%%			Vinicius Cardoso Garcia (vinicius.garcia@gmail.com)
%%
%%          Thanks to RiSE's reasearchers contributions.
%%
%% This document is based on UFPEThesis template, from Paulo Gustavo S. Fonseca.
%%
%%
%% Options:
%%         * Idiom
%%           pt   - Portguese (default)
%%           en   - English
%%
%%         * Text type
%%           bsc  - B.Sc. Thesis
%%           msc  - M.Sc. Thesis (default)
%%           qual - PHD qualification
%%           prop - PHD proposal
%%           phd  - PHD thsis
%%
%%         * Media
%%           scr  - to eletronic version (PDF) / see the users guide
%%
%%
%%         * Pagination
%%           oneside - unique face press
%%           twoside - two faces press
%%
%% Reference commands. Use the following commands to make references in your
%% text:
%%			\figref  -- for Figure reference
%% 			\tabref  -- for Table reference
%% 			\eqnref  -- for equation reference
%% 			\chapref -- for chapter reference
%% 			\secref  -- for section reference
%% 			\appref  -- for appendix reference
%% 			\axiref  -- for axiom reference
%% 			\conjref -- for conjecture reference
%% 			\defref  -- for definition reference
%% 			\lemref  -- for lemma reference
%% 			\theoref -- for theorem reference
%% 			\corref  -- for corollary reference
%% 			\propref -- for proprosition reference
%% 			\pgref   -- for page reference
%%
%%			Example: See \chapref{chap:introduction}

\documentclass[en,oneside]{risethesis}

\usepackage{natbib}
\usepackage{babel}
\usepackage{supertabular}
\usepackage{fancybox}
\usepackage{acronym}
\usepackage[linkcolor=blue,citecolor=blue,urlcolor=blue,colorlinks,pdfpagelabels,pdftitle={ycc-msc},pdfauthor={Yguarat�
Cerqueira Cavalcanti}]{hyperref}

\address{RECIFE}
\department{Centro de Inform�tica}
\program{P�s-gradua��o em Ci�ncia da Computa��o}
\majorfield{Ci�ncia da Computa��o}
\title{A Bug Report Analysis and Search Tool}
\date{JULY/2009}
\author{Yguarat� Cerqueira Cavalcanti}
\adviser{Silvio Romero de Lemos Meira}
\coadviser{Eduardo Santana de Almeida}

\begin{document}

\frontmatter
\frontpage
\presentationpage

\begin{dedicatory}    
I dedicate this dissertation to myself and all my family, friends and professors
who gave me all necessary support to get here.
\end{dedicatory}

\acknowledgements
I would like to thank and dedicate this dissertation to the following people:
\ldots

\begin{epigraph}[Open My Eyes]{S.o.ja.}
I open my eyes each morning I rise, to find a true\\
thought, know that it's real, I'm lucky to breathe,\\
I'm lucky to feel, I'm glad to wake up, I'm glad to be\\
here, with all of this world, and all of it's pain, all\\
of it's lies, and all of it's flipped down, I still\\
feel a sense of freedom, so glad I'm around,\\
\vspace{0.5cm}
It's my freedom, can't take it from me, i know it, it\\
won't change, but we need some understanding, I know\\
we'll be all right.
\end{epigraph}

\resumo
Manuten��o e evolu��o de \emph{software} s�o atividades caracterizadas pelo
seu enorme custo e baixa velocidade de execu��o. N�o obstante, elas s�o
atividades inevit�veis para garantir a qualidade do \emph{software} -- quase todo
\emph{software} bem sucedido estimula os usu�rios a fazer pedidos de mudan�as
e melhorias. Sommerville � ainda mais enf�tico e diz que mudan�as em projetos de
\emph{software} s�o um fato. Al�m disso, diferentes estudos t�m afirmado ao
longo dos anos que as atividades de manuten��o e evolu��o de \emph{software} s�o
as mais caras do ciclo de desenvolvimento, sendo respons�vel por cerca de
at� 90\% dos custos.

Todas essas peculiaridades da fase de manuten��o e evolu��o de \emph{software}
leva o mundo acad�mico e industrial a investigar constantemente novas
solu��es para reduzir os custos dessas atividades. Neste contexto, Ger�ncia de
Configura��o de Software (GCS) � um conjunto de atividades e normas para a
gest�o da evolu��o e manuten��o de \emph{software}; GCS define como s�o
registradas e processadas todas as modifica��es, o impacto das mesmas em todo
o sistema, dentre outros procedimentos. Para todas estas tarefas de GCM
existem diferentes ferramentas de aux�lio, tais como sistemas de controle de
vers�o e \emph{bug trackers}. No entanto, alguns problemas podem surgir devido ao uso
das mesmas, como por exemplo o problema de atribui��o autom�tica de respons�vel
por um \emph{bug report} e o problema de duplica��o de \emph{bug reports}.

Neste sentido, esta disserta��o investiga o problema de duplica��o de
\emph{bug reports} resultante da utiliza��o de \emph{bug trackers} em projetos
de desenvolvimento de \emph{software}. Tal problema � caracterizado pela
submiss�o de dois ou mais \emph{bug reports} que descrevem o mesmo problema
referente a um \emph{software}, tendo como principais conseq��ncias a
sobrecarga de trabalho na busca e an�lise de \emph{bug reports}, e o mal
aproveitamento do tempo destinado a essa atividade.

\begin{keywords}
relatos de bug, gerenciadores de relatos de bug, relatos de bug duplicados,
requisi��o de mudan�a, experimento, estudo de caracteriza��o, ferramenta, busca
\end{keywords}

\abstract
The efficient management of \acp{cr} is fundamental for successful software
maintenance; however the assignment of \acp{cr} to developers is an expensive
aspects in this regard, due to the time and expertise demanded. To overcome
this, researchers have proposed automated approaches for \ac{cr} assignment.
Although these proposals present advances to this topic, they do not consider
many factors inherent to the assignments. Indeed, different complex factors may
have influence on \ac{cr} assignment, and most of them vary from one
organization to another. For instance, developers' workload, \acp{cr} severity,
interpersonal relationships, or developers know-how must be considered in the
assignments. Actually, as we demonstrate in this work, \ac{cr} assignment is a
complex activity and automated approaches cannot rely on simplistic solutions.
Ideally, it is necessary to consider and reason over contextual information in
order to provide an effective automation.

In this regarding, this work proposes, implements, and validates a context-aware
architecture to automate \ac{cr} assignment. The architecture emphasizes the
need for considering the different information available at the organization to
provide a more context-aware solution to automated \ac{cr} assignment. The
development of such architecture is supported by evidence synthesized from two
empirical studies: a survey with practitioners and a systematic mapping study.
The survey provided us with a set of requirements that automated approaches
should satisfy. In the mapping study, in turn, we figured out how
state-of-the-art approaches are implemented in regarding to these requirements,
concluding that many of them are not satisfied. In addition, new requirements
were identified in this mapping study.

For the implementation of the proposed architecture, we developed a strategy to
automate \ac{cr} assignments which is based on two main components: a \acf{rbes}
and an \acf{ir} model. The strategy coordinately applies these two components in
different steps to find the potential developer to a \ac{cr}. The \ac{rbes}
takes care of the simple and complex rules necessary to consider contextual
information in the assignments, e.g., to prevent assigning a \ac{cr} to a busy
or unavailable developer. Since these rules vary from one organization/project
to another, the \ac{rbes} facilitates their modification for different contexts.
On the other hand, the \ac{ir} model is useful to make use of the historical
information of \ac{cr} assignments to match \acp{cr} and developers.

\begin{keywords}
Software Engineering, Software Maintenance and Evolution, Change Request
Management, Automatic Change Request Assignment
\end{keywords}

\tableofcontents
\listoffigures
\listoftables
\mainmatter

% Acronyms
\begin{acronym}[ACRONYM] 
% Change the word ACRONYM above to change the acronym column width.
% The column width is equals to the width of the word that you put.
% Read the manual about acronym package for more examples:
%   http://linorg.usp.br/CTAN/macros/latex/contrib/acronym/acronym.pdf
  \acro{AJAX}{Asynchronous JavaScript and XML}
  \acro{BAST}{Bug Report Analysis and Search Tool}
  \acro{BTT}{Bug Report Tracker Tool}
  \acro{BRN}{Bug Report Network}
  \acro{CCB}{Change Control Board}
\end{acronym}


%% RiSE's latex template for thesis and dissertations
%% Version 0.2
%% (c) 2008 Yguarat� Cerqueira Cavalcanti (yguarata@gmail.com)
%%			Vinicius Cardoso Garcia (vinicius.garcia@gmail.com)
%%
%%          Thanks to RiSE's reasearchers contributions.
%%
%% This document is based on UFPEThesis template, from Paulo Gustavo S. Fonseca.
%%
%%
%% Options:
%%         * Idiom
%%           pt   - Portguese (default)
%%           en   - English
%%
%%         * Text type
%%           bsc  - B.Sc. Thesis
%%           msc  - M.Sc. Thesis (default)
%%           qual - PHD qualification
%%           prop - PHD proposal
%%           phd  - PHD thsis
%%
%%         * Media
%%           scr  - to eletronic version (PDF) / see the users guide
%%
%%
%%         * Pagination
%%           oneside - unique face press
%%           twoside - two faces press
%%
%% Reference commands. Use the following commands to make references in your
%% text:
%%			\figref  -- for Figure reference
%% 			\tabref  -- for Table reference
%% 			\eqnref  -- for equation reference
%% 			\chapref -- for chapter reference
%% 			\secref  -- for section reference
%% 			\appref  -- for appendix reference
%% 			\axiref  -- for axiom reference
%% 			\conjref -- for conjecture reference
%% 			\defref  -- for definition reference
%% 			\lemref  -- for lemma reference
%% 			\theoref -- for theorem reference
%% 			\corref  -- for corollary reference
%% 			\propref -- for proprosition reference
%% 			\pgref   -- for page reference
%%
%%			Example: See \chapref{chap:introduction}

\documentclass[en,oneside]{risethesis}

\usepackage{natbib}
\usepackage{babel}
\usepackage{supertabular}
\usepackage{fancybox}
\usepackage{acronym}
\usepackage[linkcolor=blue,citecolor=blue,urlcolor=blue,colorlinks,pdfpagelabels,pdftitle={ycc-msc},pdfauthor={Yguarat�
Cerqueira Cavalcanti}]{hyperref}

\address{RECIFE}
\department{Centro de Inform�tica}
\program{P�s-gradua��o em Ci�ncia da Computa��o}
\majorfield{Ci�ncia da Computa��o}
\title{A Bug Report Analysis and Search Tool}
\date{JULY/2009}
\author{Yguarat� Cerqueira Cavalcanti}
\adviser{Silvio Romero de Lemos Meira}
\coadviser{Eduardo Santana de Almeida}

\begin{document}

\frontmatter
\frontpage
\presentationpage

\begin{dedicatory}    
I dedicate this dissertation to myself and all my family, friends and professors
who gave me all necessary support to get here.
\end{dedicatory}

\acknowledgements
I would like to thank and dedicate this dissertation to the following people:
\ldots

\begin{epigraph}[Open My Eyes]{S.o.ja.}
I open my eyes each morning I rise, to find a true\\
thought, know that it's real, I'm lucky to breathe,\\
I'm lucky to feel, I'm glad to wake up, I'm glad to be\\
here, with all of this world, and all of it's pain, all\\
of it's lies, and all of it's flipped down, I still\\
feel a sense of freedom, so glad I'm around,\\
\vspace{0.5cm}
It's my freedom, can't take it from me, i know it, it\\
won't change, but we need some understanding, I know\\
we'll be all right.
\end{epigraph}

\resumo
Manuten��o e evolu��o de \emph{software} s�o atividades caracterizadas pelo
seu enorme custo e baixa velocidade de execu��o. N�o obstante, elas s�o
atividades inevit�veis para garantir a qualidade do \emph{software} -- quase todo
\emph{software} bem sucedido estimula os usu�rios a fazer pedidos de mudan�as
e melhorias. Sommerville � ainda mais enf�tico e diz que mudan�as em projetos de
\emph{software} s�o um fato. Al�m disso, diferentes estudos t�m afirmado ao
longo dos anos que as atividades de manuten��o e evolu��o de \emph{software} s�o
as mais caras do ciclo de desenvolvimento, sendo respons�vel por cerca de
at� 90\% dos custos.

Todas essas peculiaridades da fase de manuten��o e evolu��o de \emph{software}
leva o mundo acad�mico e industrial a investigar constantemente novas
solu��es para reduzir os custos dessas atividades. Neste contexto, Ger�ncia de
Configura��o de Software (GCS) � um conjunto de atividades e normas para a
gest�o da evolu��o e manuten��o de \emph{software}; GCS define como s�o
registradas e processadas todas as modifica��es, o impacto das mesmas em todo
o sistema, dentre outros procedimentos. Para todas estas tarefas de GCM
existem diferentes ferramentas de aux�lio, tais como sistemas de controle de
vers�o e \emph{bug trackers}. No entanto, alguns problemas podem surgir devido ao uso
das mesmas, como por exemplo o problema de atribui��o autom�tica de respons�vel
por um \emph{bug report} e o problema de duplica��o de \emph{bug reports}.

Neste sentido, esta disserta��o investiga o problema de duplica��o de
\emph{bug reports} resultante da utiliza��o de \emph{bug trackers} em projetos
de desenvolvimento de \emph{software}. Tal problema � caracterizado pela
submiss�o de dois ou mais \emph{bug reports} que descrevem o mesmo problema
referente a um \emph{software}, tendo como principais conseq��ncias a
sobrecarga de trabalho na busca e an�lise de \emph{bug reports}, e o mal
aproveitamento do tempo destinado a essa atividade.

\begin{keywords}
relatos de bug, gerenciadores de relatos de bug, relatos de bug duplicados,
requisi��o de mudan�a, experimento, estudo de caracteriza��o, ferramenta, busca
\end{keywords}

\abstract
The efficient management of \acp{cr} is fundamental for successful software
maintenance; however the assignment of \acp{cr} to developers is an expensive
aspects in this regard, due to the time and expertise demanded. To overcome
this, researchers have proposed automated approaches for \ac{cr} assignment.
Although these proposals present advances to this topic, they do not consider
many factors inherent to the assignments. Indeed, different complex factors may
have influence on \ac{cr} assignment, and most of them vary from one
organization to another. For instance, developers' workload, \acp{cr} severity,
interpersonal relationships, or developers know-how must be considered in the
assignments. Actually, as we demonstrate in this work, \ac{cr} assignment is a
complex activity and automated approaches cannot rely on simplistic solutions.
Ideally, it is necessary to consider and reason over contextual information in
order to provide an effective automation.

In this regarding, this work proposes, implements, and validates a context-aware
architecture to automate \ac{cr} assignment. The architecture emphasizes the
need for considering the different information available at the organization to
provide a more context-aware solution to automated \ac{cr} assignment. The
development of such architecture is supported by evidence synthesized from two
empirical studies: a survey with practitioners and a systematic mapping study.
The survey provided us with a set of requirements that automated approaches
should satisfy. In the mapping study, in turn, we figured out how
state-of-the-art approaches are implemented in regarding to these requirements,
concluding that many of them are not satisfied. In addition, new requirements
were identified in this mapping study.

For the implementation of the proposed architecture, we developed a strategy to
automate \ac{cr} assignments which is based on two main components: a \acf{rbes}
and an \acf{ir} model. The strategy coordinately applies these two components in
different steps to find the potential developer to a \ac{cr}. The \ac{rbes}
takes care of the simple and complex rules necessary to consider contextual
information in the assignments, e.g., to prevent assigning a \ac{cr} to a busy
or unavailable developer. Since these rules vary from one organization/project
to another, the \ac{rbes} facilitates their modification for different contexts.
On the other hand, the \ac{ir} model is useful to make use of the historical
information of \ac{cr} assignments to match \acp{cr} and developers.

\begin{keywords}
Software Engineering, Software Maintenance and Evolution, Change Request
Management, Automatic Change Request Assignment
\end{keywords}

\tableofcontents
\listoffigures
\listoftables
\mainmatter

% Acronyms
\begin{acronym}[ACRONYM] 
% Change the word ACRONYM above to change the acronym column width.
% The column width is equals to the width of the word that you put.
% Read the manual about acronym package for more examples:
%   http://linorg.usp.br/CTAN/macros/latex/contrib/acronym/acronym.pdf
  \acro{AJAX}{Asynchronous JavaScript and XML}
  \acro{BAST}{Bug Report Analysis and Search Tool}
  \acro{BTT}{Bug Report Tracker Tool}
  \acro{BRN}{Bug Report Network}
  \acro{CCB}{Change Control Board}
\end{acronym}


%% RiSE's latex template for thesis and dissertations
%% Version 0.2
%% (c) 2008 Yguarat� Cerqueira Cavalcanti (yguarata@gmail.com)
%%			Vinicius Cardoso Garcia (vinicius.garcia@gmail.com)
%%
%%          Thanks to RiSE's reasearchers contributions.
%%
%% This document is based on UFPEThesis template, from Paulo Gustavo S. Fonseca.
%%
%%
%% Options:
%%         * Idiom
%%           pt   - Portguese (default)
%%           en   - English
%%
%%         * Text type
%%           bsc  - B.Sc. Thesis
%%           msc  - M.Sc. Thesis (default)
%%           qual - PHD qualification
%%           prop - PHD proposal
%%           phd  - PHD thsis
%%
%%         * Media
%%           scr  - to eletronic version (PDF) / see the users guide
%%
%%
%%         * Pagination
%%           oneside - unique face press
%%           twoside - two faces press
%%
%% Reference commands. Use the following commands to make references in your
%% text:
%%			\figref  -- for Figure reference
%% 			\tabref  -- for Table reference
%% 			\eqnref  -- for equation reference
%% 			\chapref -- for chapter reference
%% 			\secref  -- for section reference
%% 			\appref  -- for appendix reference
%% 			\axiref  -- for axiom reference
%% 			\conjref -- for conjecture reference
%% 			\defref  -- for definition reference
%% 			\lemref  -- for lemma reference
%% 			\theoref -- for theorem reference
%% 			\corref  -- for corollary reference
%% 			\propref -- for proprosition reference
%% 			\pgref   -- for page reference
%%
%%			Example: See \chapref{chap:introduction}

\documentclass[en,oneside]{risethesis}

\usepackage{natbib}
\usepackage{babel}
\usepackage{supertabular}
\usepackage{fancybox}
\usepackage{acronym}
\usepackage[linkcolor=blue,citecolor=blue,urlcolor=blue,colorlinks,pdfpagelabels,pdftitle={ycc-msc},pdfauthor={Yguarat�
Cerqueira Cavalcanti}]{hyperref}

\address{RECIFE}
\department{Centro de Inform�tica}
\program{P�s-gradua��o em Ci�ncia da Computa��o}
\majorfield{Ci�ncia da Computa��o}
\title{A Bug Report Analysis and Search Tool}
\date{JULY/2009}
\author{Yguarat� Cerqueira Cavalcanti}
\adviser{Silvio Romero de Lemos Meira}
\coadviser{Eduardo Santana de Almeida}

\begin{document}

\frontmatter
\frontpage
\presentationpage

\begin{dedicatory}    
I dedicate this dissertation to myself and all my family, friends and professors
who gave me all necessary support to get here.
\end{dedicatory}

\acknowledgements
I would like to thank and dedicate this dissertation to the following people:
\ldots

\begin{epigraph}[Open My Eyes]{S.o.ja.}
I open my eyes each morning I rise, to find a true\\
thought, know that it's real, I'm lucky to breathe,\\
I'm lucky to feel, I'm glad to wake up, I'm glad to be\\
here, with all of this world, and all of it's pain, all\\
of it's lies, and all of it's flipped down, I still\\
feel a sense of freedom, so glad I'm around,\\
\vspace{0.5cm}
It's my freedom, can't take it from me, i know it, it\\
won't change, but we need some understanding, I know\\
we'll be all right.
\end{epigraph}

\resumo
\input{resumo}

\abstract
\input{abstract}

\tableofcontents
\listoffigures
\listoftables
\mainmatter

% Acronyms
\input{acronyms}

\include{chapters/introduction/main}

\bibliographystyle{natbib}
\addcontentsline{toc}{chapter}{Bibliography}
\bibliography{references}

% Appendix
\clearpage
\addappheadtotoc
\appendix
\appendixpage
\include{appendix/experiment-instruments}

\end{document}


\bibliographystyle{natbib}
\addcontentsline{toc}{chapter}{Bibliography}
\bibliography{references}

% Appendix
\clearpage
\addappheadtotoc
\appendix
\appendixpage
\chapter{Experiment Instruments}
\label{ap:measurement-instruments}


\def \tick{
$[$\hspace{0.3cm}$]$
}

\def \twooption#1#2{
\tick #1.  \tick #2.
}

\def \threeoption#1#2#3{
\tick #1.\newline
\tick #2.\newline
\tick #3.
}

\def \fouroption#1#2#3#4{
\tick #1.\newline
\tick #2.\newline
\tick #3.\newline
\tick #4.
}

\def \fiveoption#1#2#3#4#5{
\tick #1.\newline
\tick #2.\newline
\tick #3.\newline
\tick #4.\newline
\tick #5.
}
\def \datefield{
%date fied used in time sheets
    /\hspace{0.4cm}/
}

\def \rcolor{
%table row color
    \rowcolor[gray]{0.9}
}

\def \hcolor{
    \rowcolor[gray]{0.7}
}

\section{Time sheet}
\label{ap:sec:time-sheet}

\begin{table*}[h]
  \centering
  \begin{tabular}{|c|c|c|c|c|p{5cm}|}
    \hline
    \hcolor ID & $Start~date$ & $Start~time$ & $End~date$ & $End~time$ & 
        $Is~it~a~duplicate?$\\
    \hline
       1 & \datefield  & : & \datefield & : & \twooption{Yes}{No} ID:\\
    \hline
       \rcolor 2 & \datefield  &  : & \datefield & : & \twooption{Yes}{No} ID:
       \\
    \hline
       3 & \datefield  &  : & \datefield & : & \twooption{Yes}{No} ID:\\
    \hline
       \rcolor 4 & \datefield  &  : & \datefield & : & \twooption{Yes}{No} ID:\\
    \hline
       5 & \datefield  &  : & \datefield & : & \twooption{Yes}{No} ID:\\
    \hline
       \rcolor 6 & \datefield  &  : & \datefield & : & \twooption{Yes}{No} ID:\\
    \hline
       7 & \datefield  &  : & \datefield & : & \twooption{Yes}{No} ID:\\
    \hline
       \rcolor 8 & \datefield  &  : & \datefield & : & \twooption{Yes}{No} ID:\\
    \hline
       9 & \datefield  &  : & \datefield & : & \twooption{Yes}{No} ID:\\
    \hline
       \rcolor 10 & \datefield  &  : & \datefield & : & \twooption{Yes}{No} ID:\\
    \hline
        \multicolumn{6}{|c|}{\ldots}\\
    \hline
       23 & \datefield  &  : & \datefield & : & \twooption{Yes}{No} ID:\\
    \hline
       \rcolor 24 & \datefield  &  : & \datefield & : & \twooption{Yes}{No} ID:\\
    \hline
       25 & \datefield  &  : & \datefield & : & \twooption{Yes}{No} ID:\\
    \hline
       \rcolor 26 & \datefield  &  : & \datefield & : & \twooption{Yes}{No} ID:\\
    \hline
       27 & \datefield  &  : & \datefield & : & \twooption{Yes}{No} ID:\\
    \hline
       \rcolor 28 & \datefield  &  : & \datefield & : & \twooption{Yes}{No} ID:\\
    \hline
       29 & \datefield  &  : & \datefield & : & \twooption{Yes}{No} ID:\\
    \hline
       \rcolor 30 & \datefield  &  : & \datefield & : & \twooption{Yes}{No} ID:\\
    \hline
       31 & \datefield  &  : & \datefield & : & \twooption{Yes}{No} ID:\\
    \hline
       \rcolor 32 & \datefield  &  : & \datefield & : & \twooption{Yes}{No} ID:\\
    \hline
  \end{tabular}
\caption{Time sheet used in the study.
%The subjects use this time
%sheet to note the start and end timestamps of a bug-report analysis, and mark
%if has been found a duplicate for the bug-report under analysis.
}
\end{table*}

\newpage
\section{Questionnaire for Subjects Profile}
\label{ap:sec:profile}

\begin{table}[h]
\centering
\resizebox{15cm}{!}{
\begin{tabular}{|p{\columnwidth}|}
    \hline
    \hcolor \textbf{\textsf{Questionnaire for Subjects Profile}}\\
    \hline
	    \textbf{How many years since graduation?}\\\\
	    \tick years.\\\\
    \hline
        \textbf{How many projects do you have participated according to the
        following categories?}\\ \\
        \tick Low complexity.\\
        \tick Medium complexity.\\
        \tick High complexity.\\\\
    \hline
        \textbf{What were the roles that you played in the projects cited before
        (developer, configuration manager, tester\ldots)?}
        \\
        \\
    \hline
        \textbf{How do you define your experience with bug-trackers?}\\\\
        \tick I never used them before.\\
        \tick I used them in every project i participated.\\
        I used them in \tick projects.\\\\
    \hline
        \textbf{Do you have used any of the following bug-trackers?}\\\\
        \tick Bugzilla. In: \tick industry  \tick academia\\
        \tick Trac. In: \tick industry  \tick academia\\
        \tick Mantis. In: \tick industry  \tick academia\\
        \tick Jyra. In: \tick industry  \tick academia\\
        \tick BSD Bug-tracker. In: \tick industry  \tick academia\\
        \tick Other: \\\\
    \hline
        \textbf{Have you performed any analysis of Firefox bug-reports before?}
        \\\\\twooption{Yes}{No}\\\\
    \hline
\end{tabular}
}
\caption{Questionnaire for bug-report submitters.}
\end{table}

\newpage
\section{Form for Qualitative Analysis}
\label{ap:sec:feedback}

\begin{table}[h]
\centering
\resizebox{14cm}{!}{
\begin{tabular}{|p{\columnwidth}|}
    \hline
    \hcolor \textbf{\textsf{Questionnaire for Qualitative Analysis}}\\
    \hline
        \textbf{Did you use any of the search filters provided by BAST?}
        \\
        \twooption{Yes}{No}
        \\
    \hline
        \textbf{Is there any search filter you think it must be present in BAST?}
        \\
        \twooption{Yes}{No}
        Cite them:
        \\
    \hline
        \textbf{Did you have any problem with the search filters usage?}
        \\
        \twooption{Yes}{No}
        Cite them:
        \\
    \hline
        \textbf{Did you use the ordering features of BAST?}
        \\
        \twooption{Yes}{No}
        \\
    \hline
        \textbf{Did you have any problem with ordering features?}
        \\
        \twooption{Yes}{No}
        Cite them:
        \\
    \hline
        \textbf{Do you think there is any other important information that
        must be present in the list of search results?}
        \\
        \twooption{Yes}{No}
        Cite them:
        \\
    \hline
        \textbf{Did you have any problem to visualize the details from some
        bug-report?} \\
        \twooption{Yes}{No}
        Cite them:
        \\
    \hline
        \textbf{Do you believe the way bug-reports details are presented was
        helpful to perform the analysis?}
        \\
        \twooption{Yes}{No}
        \\
    \hline 
        \textbf{Was the recommendation of related bug-reports, presented  in
        the bug-report details, useful for the analysis?}
        \\
        \twooption{Yes}{No}
        \\
    \hline
        \textbf{Is there any other information concerning bug-reports details
        you believe it should be present or emphasized?}
        \\
        \twooption{Yes}{No}
        Cite them:
        \\\\
    \hline
        \textbf{Did you use the help provided by BAST?}
        \\
        \twooption{Yes}{No}
        \\
    \hline
        \textbf{Did you found any other problem/enhacement/defect that was not
        mentioned before? Cite them.}
        \\
        \\
        \\
    \hline
        \textbf{Please, write down any suggestion you think might would be useful.}
        \\
        \\
        \\
    \hline
\end{tabular}
}
\caption{Questionnaire for qualitative analysis.}
\end{table}

\end{document}


\bibliographystyle{natbib}
\addcontentsline{toc}{chapter}{Bibliography}
\bibliography{references}

% Appendix
\clearpage
\addappheadtotoc
\appendix
\appendixpage
\chapter{Experiment Instruments}
\label{ap:measurement-instruments}


\def \tick{
$[$\hspace{0.3cm}$]$
}

\def \twooption#1#2{
\tick #1.  \tick #2.
}

\def \threeoption#1#2#3{
\tick #1.\newline
\tick #2.\newline
\tick #3.
}

\def \fouroption#1#2#3#4{
\tick #1.\newline
\tick #2.\newline
\tick #3.\newline
\tick #4.
}

\def \fiveoption#1#2#3#4#5{
\tick #1.\newline
\tick #2.\newline
\tick #3.\newline
\tick #4.\newline
\tick #5.
}
\def \datefield{
%date fied used in time sheets
    /\hspace{0.4cm}/
}

\def \rcolor{
%table row color
    \rowcolor[gray]{0.9}
}

\def \hcolor{
    \rowcolor[gray]{0.7}
}

\section{Time sheet}
\label{ap:sec:time-sheet}

\begin{table*}[h]
  \centering
  \begin{tabular}{|c|c|c|c|c|p{5cm}|}
    \hline
    \hcolor ID & $Start~date$ & $Start~time$ & $End~date$ & $End~time$ & 
        $Is~it~a~duplicate?$\\
    \hline
       1 & \datefield  & : & \datefield & : & \twooption{Yes}{No} ID:\\
    \hline
       \rcolor 2 & \datefield  &  : & \datefield & : & \twooption{Yes}{No} ID:
       \\
    \hline
       3 & \datefield  &  : & \datefield & : & \twooption{Yes}{No} ID:\\
    \hline
       \rcolor 4 & \datefield  &  : & \datefield & : & \twooption{Yes}{No} ID:\\
    \hline
       5 & \datefield  &  : & \datefield & : & \twooption{Yes}{No} ID:\\
    \hline
       \rcolor 6 & \datefield  &  : & \datefield & : & \twooption{Yes}{No} ID:\\
    \hline
       7 & \datefield  &  : & \datefield & : & \twooption{Yes}{No} ID:\\
    \hline
       \rcolor 8 & \datefield  &  : & \datefield & : & \twooption{Yes}{No} ID:\\
    \hline
       9 & \datefield  &  : & \datefield & : & \twooption{Yes}{No} ID:\\
    \hline
       \rcolor 10 & \datefield  &  : & \datefield & : & \twooption{Yes}{No} ID:\\
    \hline
        \multicolumn{6}{|c|}{\ldots}\\
    \hline
       23 & \datefield  &  : & \datefield & : & \twooption{Yes}{No} ID:\\
    \hline
       \rcolor 24 & \datefield  &  : & \datefield & : & \twooption{Yes}{No} ID:\\
    \hline
       25 & \datefield  &  : & \datefield & : & \twooption{Yes}{No} ID:\\
    \hline
       \rcolor 26 & \datefield  &  : & \datefield & : & \twooption{Yes}{No} ID:\\
    \hline
       27 & \datefield  &  : & \datefield & : & \twooption{Yes}{No} ID:\\
    \hline
       \rcolor 28 & \datefield  &  : & \datefield & : & \twooption{Yes}{No} ID:\\
    \hline
       29 & \datefield  &  : & \datefield & : & \twooption{Yes}{No} ID:\\
    \hline
       \rcolor 30 & \datefield  &  : & \datefield & : & \twooption{Yes}{No} ID:\\
    \hline
       31 & \datefield  &  : & \datefield & : & \twooption{Yes}{No} ID:\\
    \hline
       \rcolor 32 & \datefield  &  : & \datefield & : & \twooption{Yes}{No} ID:\\
    \hline
  \end{tabular}
\caption{Time sheet used in the study.
%The subjects use this time
%sheet to note the start and end timestamps of a bug-report analysis, and mark
%if has been found a duplicate for the bug-report under analysis.
}
\end{table*}

\newpage
\section{Questionnaire for Subjects Profile}
\label{ap:sec:profile}

\begin{table}[h]
\centering
\resizebox{15cm}{!}{
\begin{tabular}{|p{\columnwidth}|}
    \hline
    \hcolor \textbf{\textsf{Questionnaire for Subjects Profile}}\\
    \hline
	    \textbf{How many years since graduation?}\\\\
	    \tick years.\\\\
    \hline
        \textbf{How many projects do you have participated according to the
        following categories?}\\ \\
        \tick Low complexity.\\
        \tick Medium complexity.\\
        \tick High complexity.\\\\
    \hline
        \textbf{What were the roles that you played in the projects cited before
        (developer, configuration manager, tester\ldots)?}
        \\
        \\
    \hline
        \textbf{How do you define your experience with bug-trackers?}\\\\
        \tick I never used them before.\\
        \tick I used them in every project i participated.\\
        I used them in \tick projects.\\\\
    \hline
        \textbf{Do you have used any of the following bug-trackers?}\\\\
        \tick Bugzilla. In: \tick industry  \tick academia\\
        \tick Trac. In: \tick industry  \tick academia\\
        \tick Mantis. In: \tick industry  \tick academia\\
        \tick Jyra. In: \tick industry  \tick academia\\
        \tick BSD Bug-tracker. In: \tick industry  \tick academia\\
        \tick Other: \\\\
    \hline
        \textbf{Have you performed any analysis of Firefox bug-reports before?}
        \\\\\twooption{Yes}{No}\\\\
    \hline
\end{tabular}
}
\caption{Questionnaire for bug-report submitters.}
\end{table}

\newpage
\section{Form for Qualitative Analysis}
\label{ap:sec:feedback}

\begin{table}[h]
\centering
\resizebox{14cm}{!}{
\begin{tabular}{|p{\columnwidth}|}
    \hline
    \hcolor \textbf{\textsf{Questionnaire for Qualitative Analysis}}\\
    \hline
        \textbf{Did you use any of the search filters provided by BAST?}
        \\
        \twooption{Yes}{No}
        \\
    \hline
        \textbf{Is there any search filter you think it must be present in BAST?}
        \\
        \twooption{Yes}{No}
        Cite them:
        \\
    \hline
        \textbf{Did you have any problem with the search filters usage?}
        \\
        \twooption{Yes}{No}
        Cite them:
        \\
    \hline
        \textbf{Did you use the ordering features of BAST?}
        \\
        \twooption{Yes}{No}
        \\
    \hline
        \textbf{Did you have any problem with ordering features?}
        \\
        \twooption{Yes}{No}
        Cite them:
        \\
    \hline
        \textbf{Do you think there is any other important information that
        must be present in the list of search results?}
        \\
        \twooption{Yes}{No}
        Cite them:
        \\
    \hline
        \textbf{Did you have any problem to visualize the details from some
        bug-report?} \\
        \twooption{Yes}{No}
        Cite them:
        \\
    \hline
        \textbf{Do you believe the way bug-reports details are presented was
        helpful to perform the analysis?}
        \\
        \twooption{Yes}{No}
        \\
    \hline 
        \textbf{Was the recommendation of related bug-reports, presented  in
        the bug-report details, useful for the analysis?}
        \\
        \twooption{Yes}{No}
        \\
    \hline
        \textbf{Is there any other information concerning bug-reports details
        you believe it should be present or emphasized?}
        \\
        \twooption{Yes}{No}
        Cite them:
        \\\\
    \hline
        \textbf{Did you use the help provided by BAST?}
        \\
        \twooption{Yes}{No}
        \\
    \hline
        \textbf{Did you found any other problem/enhacement/defect that was not
        mentioned before? Cite them.}
        \\
        \\
        \\
    \hline
        \textbf{Please, write down any suggestion you think might would be useful.}
        \\
        \\
        \\
    \hline
\end{tabular}
}
\caption{Questionnaire for qualitative analysis.}
\end{table}

\end{document}


\bibliographystyle{natbib}
\addcontentsline{toc}{chapter}{\bibliographytocname}
\bibliography{references}

% Appendix
\clearpage
\addappheadtotoc
\appendix
\appendixpage
\chapter{Experiment Instruments}
\label{ap:measurement-instruments}


\def \tick{
$[$\hspace{0.3cm}$]$
}

\def \twooption#1#2{
\tick #1.  \tick #2.
}

\def \threeoption#1#2#3{
\tick #1.\newline
\tick #2.\newline
\tick #3.
}

\def \fouroption#1#2#3#4{
\tick #1.\newline
\tick #2.\newline
\tick #3.\newline
\tick #4.
}

\def \fiveoption#1#2#3#4#5{
\tick #1.\newline
\tick #2.\newline
\tick #3.\newline
\tick #4.\newline
\tick #5.
}
\def \datefield{
%date fied used in time sheets
    /\hspace{0.4cm}/
}

\def \rcolor{
%table row color
    \rowcolor[gray]{0.9}
}

\def \hcolor{
    \rowcolor[gray]{0.7}
}

\section{Time sheet}
\label{ap:sec:time-sheet}

\begin{table*}[h]
  \centering
  \begin{tabular}{|c|c|c|c|c|p{5cm}|}
    \hline
    \hcolor ID & $Start~date$ & $Start~time$ & $End~date$ & $End~time$ & 
        $Is~it~a~duplicate?$\\
    \hline
       1 & \datefield  & : & \datefield & : & \twooption{Yes}{No} ID:\\
    \hline
       \rcolor 2 & \datefield  &  : & \datefield & : & \twooption{Yes}{No} ID:
       \\
    \hline
       3 & \datefield  &  : & \datefield & : & \twooption{Yes}{No} ID:\\
    \hline
       \rcolor 4 & \datefield  &  : & \datefield & : & \twooption{Yes}{No} ID:\\
    \hline
       5 & \datefield  &  : & \datefield & : & \twooption{Yes}{No} ID:\\
    \hline
       \rcolor 6 & \datefield  &  : & \datefield & : & \twooption{Yes}{No} ID:\\
    \hline
       7 & \datefield  &  : & \datefield & : & \twooption{Yes}{No} ID:\\
    \hline
       \rcolor 8 & \datefield  &  : & \datefield & : & \twooption{Yes}{No} ID:\\
    \hline
       9 & \datefield  &  : & \datefield & : & \twooption{Yes}{No} ID:\\
    \hline
       \rcolor 10 & \datefield  &  : & \datefield & : & \twooption{Yes}{No} ID:\\
    \hline
        \multicolumn{6}{|c|}{\ldots}\\
    \hline
       23 & \datefield  &  : & \datefield & : & \twooption{Yes}{No} ID:\\
    \hline
       \rcolor 24 & \datefield  &  : & \datefield & : & \twooption{Yes}{No} ID:\\
    \hline
       25 & \datefield  &  : & \datefield & : & \twooption{Yes}{No} ID:\\
    \hline
       \rcolor 26 & \datefield  &  : & \datefield & : & \twooption{Yes}{No} ID:\\
    \hline
       27 & \datefield  &  : & \datefield & : & \twooption{Yes}{No} ID:\\
    \hline
       \rcolor 28 & \datefield  &  : & \datefield & : & \twooption{Yes}{No} ID:\\
    \hline
       29 & \datefield  &  : & \datefield & : & \twooption{Yes}{No} ID:\\
    \hline
       \rcolor 30 & \datefield  &  : & \datefield & : & \twooption{Yes}{No} ID:\\
    \hline
       31 & \datefield  &  : & \datefield & : & \twooption{Yes}{No} ID:\\
    \hline
       \rcolor 32 & \datefield  &  : & \datefield & : & \twooption{Yes}{No} ID:\\
    \hline
  \end{tabular}
\caption{Time sheet used in the study.
%The subjects use this time
%sheet to note the start and end timestamps of a bug-report analysis, and mark
%if has been found a duplicate for the bug-report under analysis.
}
\end{table*}

\newpage
\section{Questionnaire for Subjects Profile}
\label{ap:sec:profile}

\begin{table}[h]
\centering
\resizebox{15cm}{!}{
\begin{tabular}{|p{\columnwidth}|}
    \hline
    \hcolor \textbf{\textsf{Questionnaire for Subjects Profile}}\\
    \hline
	    \textbf{How many years since graduation?}\\\\
	    \tick years.\\\\
    \hline
        \textbf{How many projects do you have participated according to the
        following categories?}\\ \\
        \tick Low complexity.\\
        \tick Medium complexity.\\
        \tick High complexity.\\\\
    \hline
        \textbf{What were the roles that you played in the projects cited before
        (developer, configuration manager, tester\ldots)?}
        \\
        \\
    \hline
        \textbf{How do you define your experience with bug-trackers?}\\\\
        \tick I never used them before.\\
        \tick I used them in every project i participated.\\
        I used them in \tick projects.\\\\
    \hline
        \textbf{Do you have used any of the following bug-trackers?}\\\\
        \tick Bugzilla. In: \tick industry  \tick academia\\
        \tick Trac. In: \tick industry  \tick academia\\
        \tick Mantis. In: \tick industry  \tick academia\\
        \tick Jyra. In: \tick industry  \tick academia\\
        \tick BSD Bug-tracker. In: \tick industry  \tick academia\\
        \tick Other: \\\\
    \hline
        \textbf{Have you performed any analysis of Firefox bug-reports before?}
        \\\\\twooption{Yes}{No}\\\\
    \hline
\end{tabular}
}
\caption{Questionnaire for bug-report submitters.}
\end{table}

\newpage
\section{Form for Qualitative Analysis}
\label{ap:sec:feedback}

\begin{table}[h]
\centering
\resizebox{14cm}{!}{
\begin{tabular}{|p{\columnwidth}|}
    \hline
    \hcolor \textbf{\textsf{Questionnaire for Qualitative Analysis}}\\
    \hline
        \textbf{Did you use any of the search filters provided by BAST?}
        \\
        \twooption{Yes}{No}
        \\
    \hline
        \textbf{Is there any search filter you think it must be present in BAST?}
        \\
        \twooption{Yes}{No}
        Cite them:
        \\
    \hline
        \textbf{Did you have any problem with the search filters usage?}
        \\
        \twooption{Yes}{No}
        Cite them:
        \\
    \hline
        \textbf{Did you use the ordering features of BAST?}
        \\
        \twooption{Yes}{No}
        \\
    \hline
        \textbf{Did you have any problem with ordering features?}
        \\
        \twooption{Yes}{No}
        Cite them:
        \\
    \hline
        \textbf{Do you think there is any other important information that
        must be present in the list of search results?}
        \\
        \twooption{Yes}{No}
        Cite them:
        \\
    \hline
        \textbf{Did you have any problem to visualize the details from some
        bug-report?} \\
        \twooption{Yes}{No}
        Cite them:
        \\
    \hline
        \textbf{Do you believe the way bug-reports details are presented was
        helpful to perform the analysis?}
        \\
        \twooption{Yes}{No}
        \\
    \hline 
        \textbf{Was the recommendation of related bug-reports, presented  in
        the bug-report details, useful for the analysis?}
        \\
        \twooption{Yes}{No}
        \\
    \hline
        \textbf{Is there any other information concerning bug-reports details
        you believe it should be present or emphasized?}
        \\
        \twooption{Yes}{No}
        Cite them:
        \\\\
    \hline
        \textbf{Did you use the help provided by BAST?}
        \\
        \twooption{Yes}{No}
        \\
    \hline
        \textbf{Did you found any other problem/enhacement/defect that was not
        mentioned before? Cite them.}
        \\
        \\
        \\
    \hline
        \textbf{Please, write down any suggestion you think might would be useful.}
        \\
        \\
        \\
    \hline
\end{tabular}
}
\caption{Questionnaire for qualitative analysis.}
\end{table}

\end{document}
